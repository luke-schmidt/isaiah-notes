\documentclass[11pt]{article}
\usepackage[margin=1in]{geometry}
\usepackage{../../styles/isaiah}
\usepackage{../../styles/components/isaiahOraclesOverview}

\begin{document}

\newpage
\begin{center}
{\Huge\bfseries Isaiah 13:1-14:27}
\end{center}
\vspace{10pt}

% Oracle Overview for Chapters 13-23
\isaiahOraclesOverview{1}

% Overview of Isaiah 13:1-14:27
\begin{overview}{Isaiah 13:1-14:27 — Overview}

\overviewsection[0]{%
\textbf{13:1-5: God Announcing Judgment on Babylon}
\begin{itemize}
    \item A \highlightaqua{hand} stretches out from a \highlightbrown{mountain} to summon a judgment that destroys the \highlightgreen{whole land}
\end{itemize}
}

\overviewsection[1]{%
\textbf{13:6-22: \highlightred{Day of the LORD} Against Babylon}
\begin{itemize}
    \item \highlightdarkred{Wrath} and \highlightdarkred{anger} bring \highlightgray{Babylon} low
    \item \highlightorange{Stars} become darkened
    \item \highlightgreen{Land} inhabited by wild creatures
\end{itemize}
}

\overviewsection[2]{%
\textbf{14:1-2: God Will Have Compassion on Israel}
\begin{itemize}
    \item God makes their masters their servants
\end{itemize}
}

\overviewsection[1]{%
\textbf{14:3-23: Trash Talk Against Babylon}
\begin{itemize}
    \item \highlightdarkred{Wrath} and \highlightdarkred{anger} bring \highlightgray{Babylon} low
    \item \highlightorange{Stars} become darkened
    \item \highlightgreen{Land} inhabited by wild creatures
\end{itemize}
}

\overviewsection[0]{%
\textbf{14:24-27: God Announcing Judgment on Assyria}
\begin{itemize}
    \item A \highlightaqua{hand} stretches out from a \highlightbrown{mountain} to summon a judgment that destroys the \highlightgreen{whole land}
\end{itemize}
}

\end{overview}

% Chapter 13:1-5
\begin{biblicaloutline}[Isaiah 13:1-5]

\subsectionheader{The Oracle Against Babylon}

\begin{versesection}{2em}
\versenum{1} The oracle\footnote{lit. "Burden"} concerning \highlightgray{Babylon} which Isaiah the son of Amoz saw.

\versenum{2} On a bare \highlightbrown{hill} raise a signal\footnote{c.f. ch 2, 5, 11. All about raising a signal on a mountain/hill};
\poetryline cry aloud to them;
wave the \highlightaqua{hand} for them to enter
\poetryline the gates of the nobles.
\versenum{3} I myself have commanded my consecrated ones,
\poetryline and have summoned my mighty men to execute my \highlightdarkred{anger},
\poetryline my proudly exulting ones.

\versenum{4} The noise of a multitude in the \highlightbrown{mountains},
\poetryline like that of many people!
A tumultuous noise of the kingdoms of nations gathered together!
\poetryline The LORD of hosts musters
\poetryline the army for battle.
\versenum{5} They come from a far country,
\poetryline from the end of heaven—
the LORD and his weapons of indignation,
\poetryline to destroy the \highlightgreen{whole land}\footnote{c.f. ch 5:26}.
\end{versesection}

\end{biblicaloutline}

{\vspace{4em}}
{\large\bfseries Why Address Babylon in Isaiah's Day?}
{\vspace{1em}}

They weren't that big of a deal. Babylon is a picture of all evil in the scriptures. Reference the tower of Babel story (Genesis 11) as the origin of this nation. Super relevant for chapters 13-14.

{\vspace{1em}}

In v4 we have the nations gathering together on a mountain just like chapter 2, except this time, there's no weapons getting destroyed here.

% Chapter 13:6-16
\begin{biblicaloutline}[Isaiah 13:6-16]

\subsectionheader{The Day of the LORD}

\begin{versesection}{2em}
\versenum{6} Wail, for \highlightred{the day of the LORD} is at hand!
\poetryline It will come as destruction from the Almighty\footnote{Wordplay of Destruction - shod; Almighty - shad-dai}.
\versenum{7} Therefore all \highlightaqua{hands} will be limp,
\poetryline every man's heart will melt,
\versenum{8} and they will be amazed.
Pangs and sorrows will take hold of them;
\poetryline they will be in pain as a woman in childbirth.
They will be amazed at one another;
\poetryline their faces will be like flames.

\versenum{9} Behold, \highlightred{the day of the LORD} comes,
\poetryline cruel, with \highlightdarkred{wrath} and \highlightdarkred{fierce anger},
to make the \highlightgreen{land} a desolation
\poetryline and to destroy its sinners from it.
\versenum{10} For the \highlightorange{stars} of the heavens and their constellations
\poetryline will not give their light;
the sun will be dark at its rising,
\poetryline and the moon will not shed its light.
\versenum{11} I will punish the world for its evil,
\poetryline and the wicked for their iniquity;
I will put an end to the pomp of the arrogant,
\poetryline and lay low the pompous pride of the ruthless.
\versenum{12} I will make man more rare than fine \highlightyellow{gold},
\poetryline and mankind than the \highlightyellow{gold} of Ophir\footnote{Part of Solomon's trade route. Every use of Ophir in the Bible is about gold.}.
\versenum{13} Therefore I will make the heavens \highlightpurple{tremble}\footnote{Note the repeated word. It's mentioned 3 times (translated "stirred up" in 14:9) in reference to the heavens, earth, and sheol.},
\poetryline and the \highlightgreen{earth} will be shaken out of its place,
at the \highlightdarkred{wrath} of the LORD of hosts
\poetryline in the day of his \highlightdarkred{fierce anger}.
\versenum{14} And like a hunted gazelle,
\poetryline or like sheep with none to gather them,
each will turn to his own people,
\poetryline and each will flee to his own \highlightgreen{land}.
\versenum{15} Whoever is found will be thrust through,
\poetryline and whoever is caught will fall by the sword.
\versenum{16} Their children also will be dashed to pieces before their eyes;
\poetryline their houses will be plundered
\poetryline and their wives ravished.
\end{versesection}

\end{biblicaloutline}

% Chapter 13:17-22
\begin{biblicaloutline}[Isaiah 13:17-22 (cont.)]

\subsectionheader{The Desolation of Babylon}

\begin{versesection}{2em}
\versenum{17} Behold, I am \highlightpurple{stirring up} the Medes against them,
\poetryline who have no regard for silver
\poetryline and do not delight in \highlightyellow{gold}.
\versenum{18} Their bows will slaughter the young men;
\poetryline they will have no mercy on the fruit of the womb;
\poetryline their eyes will not pity children.
\versenum{19} And \highlightgray{Babylon}, the glory of kingdoms,
\poetryline the splendor and pomp of the Chaldeans,
will be like Sodom and Gomorrah
\poetryline when God overthrew them.
\versenum{20} It will never be inhabited
\poetryline or lived in for all generations;
no Arab will pitch his tent there;
\poetryline no shepherds will make their flocks lie down there.
\versenum{21} But wild beasts\footnote{v21-22 are like a distorted inversion of chapter 11} will lie down there,
\poetryline and their houses will be full of howling creatures;
there ostriches will dwell,
\poetryline and there wild goats will dance.
\versenum{22} Hyenas will cry in its towers,
\poetryline and jackals in the pleasant palaces;
its time is close at hand
\poetryline and its days will not be prolonged.
\end{versesection}

\end{biblicaloutline}

{\vspace{4em}}
{\large\bfseries Cosmic Decreation}
{\vspace{1em}}

Whatever is going on here doesn't seem \textit{only} isolated to Babylon, even if they are the primary audience. There's a cosmic scale of undoing Genesis 1 - land desolation, sun moon and stars losing light, etc.

{\vspace{2em}}
{\large\bfseries Cruelty?}
{\vspace{1em}}

Reading v16 makes me personally a little uncomfortable knowing it's Yahweh behind these things, but that's no different of a problem of evil today. Either God is in control or He's not. Here Isaiah highlights that Yahweh is the one "stirring them up", but He is not to blame for the atrocities that the Medes will be doing here. Isaiah (nor any other biblical author) never exposits too much on how this tension can be resolved, but just say along with Joseph, "What you meant for evil, God meant for good"

{\vspace{2em}}
{\large\bfseries No Mercy?}
{\vspace{1em}}

Verses 17-22 didn't happen literally like described here as Babylon ended up surrendering without a fight to Cyrus and the Persians (who took over the Medes). This is more representative of the evil beneath Babylon and all the world powers against God.

% Chapter 14:1-2
\begin{biblicaloutline}[Isaiah 14:1-2]

\subsectionheader{Mercy on Israel}

\begin{versesection}{2em}
\versenum{1} For the LORD will have compassion on Jacob and will again choose Israel, and will set them in their own \highlightgreen{land}, and sojourners will join them and will attach themselves to the house of Jacob.
\versenum{2} And the peoples will take them and bring them to their place, and the house of Israel will possess them in the \highlightgreen{LORD's land} as male and female slaves. They will take captive those who were their captors, and rule over those who oppressed them.
\end{versesection}

\end{biblicaloutline}

{\vspace{4em}}
{\large\bfseries Rulers in Their Land}
{\vspace{1em}}

Verses 1-2 were fulfilled in the short term in Ezra 1:1-4 where the survivors were assisted by the nations with gold and such. In the long term, we see a fulfillment in Matthew 5:5 and Revelation 5:9-10. In the garden, Adam and Eve in Genesis 1:28 were called to subdue the earth and rule it - God's people again will do so in eternity.

\begin{quote}
\textit{"In the first year of Cyrus king of Persia, that the word of the LORD by the mouth of Jeremiah might be fulfilled, the LORD stirred up the spirit of Cyrus king of Persia, so that he made a proclamation throughout all his kingdom and also put it in writing: 'Thus says Cyrus king of Persia: The LORD, the God of heaven, has given me all the kingdoms of the earth, and he has charged me to build him a house at Jerusalem, which is in Judah. Whoever is among you of all his people, may his God be with him, and let him go up to Jerusalem, which is in Judah, and rebuild the house of the LORD, the God of Israel—he is the God who is in Jerusalem. And let each survivor, in whatever place he sojourns, be assisted by the men of his place with silver and gold, with goods and with beasts, besides freewill offerings for the house of God that is in Jerusalem.'"}\\
\hfill --- Ezra 1:1-4
\end{quote}

\begin{quote}
\textit{"Blessed are the meek, for they shall inherit the earth."}\\
\hfill --- Matthew 5:5
\end{quote}

\begin{quote}
\textit{"And they sang a new song, saying, 'Worthy are you to take the scroll and to open its seals, for you were slain, and by your blood you ransomed people for God from every tribe and language and people and nation, and you have made them a kingdom and priests to our God, and they shall reign on the earth.'"}\\
\hfill --- Revelation 5:9-10
\end{quote}

% Chapter 14:3-15
\begin{biblicaloutline}[Isaiah 14:3-15]

\subsectionheader{Taunt Against the King of Babylon}

\begin{versesection}{2em}
\versenum{3} When the LORD has given you rest\footnote{Genesis 2:15} from your pain\footnote{Genesis 3:16-17} and turmoil and the hard service\footnote{Exodus 1:14} with which you were made to serve,
\versenum{4} you will take up this taunt against the king of \highlightgray{Babylon}:

"How the oppressor has ceased,
\poetryline the insolent fury ceased!
\versenum{5} The LORD has broken the staff of the wicked,
\poetryline the scepter of rulers,
\versenum{6} that struck the peoples in \highlightdarkred{wrath}
\poetryline with unceasing blows,
that ruled the nations in \highlightdarkred{anger}
\poetryline with unrelenting persecution.
\versenum{7} The \highlightgreen{whole earth} is at rest and quiet;
\poetryline they break forth into singing.
\versenum{8} The cypress trees rejoice at you,
\poetryline the cedars of Lebanon, saying,
'Since you were laid low,
\poetryline no woodcutter comes up against us.'

\versenum{9} Sheol beneath is \highlightpurple{stirred up}
\poetryline to meet you when you come;
it rouses the shades\footnote{The "Rephaim" - name for early Canaanite dwellers and likely related to the Nephilim of Genesis 6} to greet you,
\poetryline all who were leaders of the \highlightgreen{earth};
it raises from their thrones
\poetryline all who were kings of the nations.
\versenum{10} All of them will answer
\poetryline and say to you:
'You too have become as weak as we!
\poetryline You have become like us!'
\versenum{11} Your pomp is brought down to Sheol,
\poetryline the sound of your harps;
maggots are laid as a bed beneath you,
\poetryline and worms are your covers.

\versenum{12} "How you are fallen from heaven,
\poetryline O Day Star, son of Dawn!
How you are cut down to the ground,
\poetryline you who laid the nations low!
\versenum{13} You said in your heart,
\poetryline 'I will ascend to heaven;
above the \highlightorange{stars} of God
\poetryline I will set my throne on high;
I will sit on the \highlightbrown{mount} of assembly
\poetryline in the far reaches of the north\footnote{Lit. "the remote parts of Zaphon" where the "gods" met};
\versenum{14} I will ascend above the heights of the clouds;
\poetryline I will make myself like the Most High.'
\versenum{15} But you are brought down to Sheol,
\poetryline to the far reaches of the pit.
\end{versesection}

\end{biblicaloutline}

% Chapter 14:16-23
\begin{biblicaloutline}[Isaiah 14:16-23 (cont.)]

\subsectionheader{The Fall of Babylon's King}

\begin{versesection}{2em}
\versenum{16} Those who see you will stare at you
\poetryline and ponder over you:
'Is this the man who made the \highlightgreen{earth} \highlightpurple{tremble},
\poetryline who shook kingdoms,
\versenum{17} who made the world like a desert
\poetryline and overthrew its cities,
\poetryline who did not let his prisoners go home\footnote{Like Pharaoh (see v3)}?'
\versenum{18} All the kings of the nations lie in glory,
\poetryline each in his own tomb;
\versenum{19} but you are cast out, away from your grave,
\poetryline like a loathed branch,
clothed with the slain, those pierced by the sword,
\poetryline who go down to the stones of the pit,
\poetryline like a dead body trampled underfoot.
\versenum{20} You will not be joined with them in burial,
\poetryline because you have destroyed your \highlightgreen{land},
\poetryline you have slain your people.

"May the offspring of evildoers
\poetryline nevermore be named!
\versenum{21} Prepare slaughter for his sons
\poetryline because of the guilt of their fathers,
lest they rise and possess the \highlightgreen{earth}
\poetryline and fill the face of the world with cities."

\versenum{22} "I will rise up against them," declares the LORD of hosts, "and will cut off from \highlightgray{Babylon} name and remnant, descendants and posterity," declares the LORD.
\versenum{23} "And I will make it a possession of the hedgehog, and pools of water, and I will sweep it with the broom of destruction," declares the LORD of hosts.
\end{versesection}

\end{biblicaloutline}

{\vspace{4em}}
{\large\bfseries Day Star, Son of Dawn}
{\vspace{1em}}

There's an interesting history with how this passage has been read back to refer to the fall of Lucifer (light-bearer, from this passage), but I think there's more going on here with how this imagery is in conversation with the stories of the Canaanite gods. The Day Star itself is likely referring to Venus where there's this last star that holds on just long enough before the sun finally rules back over it. This is also in conversation with the ancient Canaanite story of how the storm god, Baal, died and then Athtar tried to take the throne on the mount of Zaphon. He couldn't though since he was too weak, so he was cast down.

{\vspace{1em}}

King of Babylon is described the same way here → Grasping for power wrongfully and being humbled in the process. This fall of Babylon is likely not so figurative to describe Satan but also not so literal to just be talking about one Babylonian ruler. It seems hand-in-hand with something like Revelation 18 where Babylon is both the nation and rule, but also (sometimes more-so) about the evil powers behind it. This is why we can rejoice at evil being destroyed and "gloat" like Israel is doing here. It's not an isolated nation, it's a biblical pattern.

{\vspace{2em}}
{\large\bfseries Fill World with Cities?}
{\vspace{1em}}

Cities were not put in the greatest light starting with the first city built by Cain and the continuation of them in the Bible. They ultimately get redeemed though in Revelation.

\begin{quote}
\textit{"In days to come Jacob shall take root, Israel shall blossom and put forth shoots and fill the whole world with fruit."}\\
\hfill --- Isaiah 27:6
\end{quote}

\begin{quote}
\textit{"And I saw the holy city, new Jerusalem, coming down out of heaven from God, prepared as a bride adorned for her husband."}\\
\hfill --- Revelation 21:2
\end{quote}

% Chapter 14:24-27
\begin{biblicaloutline}[Isaiah 14:24-27]

\subsectionheader{Judgment on Assyria}

\begin{versesection}{2em}
\versenum{24} The LORD of hosts has sworn:
"As I have planned,
\poetryline so shall it be,
and as I have purposed,
\poetryline so shall it stand,
\versenum{25} that I will break the Assyrian in my \highlightgreen{land},
\poetryline and on my \highlightbrown{mountains} trample him underfoot;
and his yoke shall depart from them,
\poetryline and his burden from their shoulder."
\versenum{26} This is the purpose that is purposed
\poetryline concerning the \highlightgreen{whole earth},
and this is the \highlightaqua{hand} that is stretched out
\poetryline over all the nations.
\versenum{27} For the LORD of hosts has purposed,
\poetryline and who will annul it?
His \highlightaqua{hand} is stretched out,
\poetryline and who will turn it back?
\end{versesection}

\end{biblicaloutline}

{\vspace{4em}}
{\large\bfseries Why Bring Assyria In On This?}
{\vspace{1em}}

What God is doing to Assyria here is what He'll do to Babylon as well.

\end{document}
