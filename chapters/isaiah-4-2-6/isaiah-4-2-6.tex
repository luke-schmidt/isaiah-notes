\documentclass[11pt]{article}
\usepackage[margin=1in]{geometry}
\usepackage{../../styles/isaiah}

\begin{document}

\begin{center}
{\Huge\bfseries Isaiah 4:2-6 – The Branch of the LORD}
\end{center}

% Overview of Isaiah 1:2-5:30 with section 4:2-6 bolded
\isaiahOverviewGrid{5}

{\vspace{3em}}


% Isaiah 4:2-6 Study
\begin{biblicaloutline}[Isaiah 4:2-6]

\subsectionheader{The Branch of the LORD (2-3)}

\begin{versesection}{2em}
\versenum{2} In that day the branch of the LORD shall be beautiful and glorious,
\poetryline and the fruit of the land shall be the pride and honor
\poetryline of the survivors of Israel.

\versenum{3} And he who is left in Zion and remains in Jerusalem will be called holy,
\poetryline everyone who has been recorded for life in Jerusalem,
\end{versesection}

\subsectionheader{Divine Cleansing and Protection (4-6)}

\begin{versesection}{2em}
\versenum{4} when the Lord shall have \highlightblue{washed away} the filth of the daughters of Zion
\poetryline and \highlightblue{cleansed} the bloodstains of Jerusalem from its midst
\poetryline by a spirit of judgment and by a spirit of \highlightred{burning}.

\versenum{5} Then the LORD will create over the whole site of Mount Zion
\poetryline and over her assemblies
\poetryline a cloud by day, and smoke and the shining of a flaming fire by night;
\poetryline for over all the glory there will be a canopy.

\versenum{6} There will be a booth for shade by day from \highlightred{the heat},
\poetryline and for a refuge and a shelter from the \highlightblue{storm and rain}.
\end{versesection}

\end{biblicaloutline}

\vspace{3em}
{\large\bfseries Sprout of Yahweh}
\vspace{1em}

\hebrewword{Sprout}{צֶ֫מַח}{TSEH-makh}{Means "branch," "shoot," or "growth." Used of vegetation sprouting from the ground (Genesis 2:9) and as a Messianic title (Jeremiah 33:15, Zechariah 6:12). This word evokes new life emerging from what seemed dead or cut off, pointing to the coming Messiah who will bring restoration.}

The word for branch here in Verse 2 literally means "sprout" or "shoot" (\hebrew{צֶ֫מַח} – tse.mach). It's the same word used in reference
to the fruit trees that sprout in Eden (Gen 2:9). This imagery evokes the idea of new life and growth emerging from what seemed dead or cut off.
\vspace{1em}

Elsewhere this term is used in the prophets as an explicit Messianic reference (Jer 33:15, Zec 6:12), but the immediate context leads some folks to see it as describing the glorious land and Eden-like abundance that the remnant will have.
\vspace{1em}

Given that Isaiah picks up on a Messianic hope for Israel and uses similar terms for Him just a few chapters later (Isaiah 11:1,10), assuming this "sprout" is purely an abundance reference may be a bit flat.


\vspace{3em}
{\large\bfseries Recorded for Life}
\vspace{1em}

Verse 3 isn't the first time we've seen in the Hebrew Bible about a scroll that Yahweh is keeping that tracks who's been recorded for life.
In Exodus 32:32-33, Moses offers to have his name blotted out of the book that Yahweh has written if it would save the people.

\begin{quote}
\textit{But now, if you will forgive their sin, please forgive it; but if not, please wipe me out of your book that you have written.
\\And the LORD said to Moses, “Whoever has sinned against me, I will blot out of my book.”}\\\\
\hfill --- Exodus 32:32-33
\end{quote}

{\vspace{1em}}

This imagery likely stems from the Ancient Near East practice of keeping a record of the living and the dead, where the king would have a "census" scroll of his subjects. To be removed from this list likely meant either exile or death.

{\vspace{1em}}

Yahweh Himself has his own scroll in which those who are the holy ones are written for life. Likely not just physical life, but a pure heart that will see the Glory of the LORD.


\vspace{3em}
{\large\bfseries Washed away...by Fire?}
\vspace{1em}

In Verse 4, after the fruitful, Eden-like land, we have another early Genesis image with a flood of judgement washing away the filth of the daughters of Zion.

{\vspace{1em}}

The only thing though is this washing away and "cleansing" (another word for washing, always used in association with water), is done by a "spirit of judgment" and a "spirit of burning"?

{\vspace{1em}}

\hebrewword{Spirit}{רוּחַ}{ROO-akh}{Can mean "wind," "breath," or "spirit." It's the invisible life-energy that animates and sustains all living things. In this context, it represents God's powerful presence working to cleanse and purify His people through both judgment and renewal.}

The word spirit (\hebrew{רוּחַ} – ru.ach) can refer to wind, breath, or spirit. It's the invisible life-energy that animates and sustains all living things.

{\vspace{1em}}

So this phrase of "spirit of burning" could also imply a fire wind of sorts that is "cleaning house" so to speak, leaving the holy remnant in Zion.


\vspace{3em}
{\large\bfseries Tabernacle Glory}
\vspace{1em}


\hebrewword{Create}{בָּרָא}{bah-RAH}{The same word used in Genesis 1:1 for God's original creation. This verb is reserved exclusively for God's creative activity, emphasizing that only God can bring something into existence from nothing. Here it points to a new creation - a new exodus and new covenant community on Mount Zion.}

On to Verse 5! The use of the word "create" (\hebrew{בָּרָא} – ba.ra) is a direct callback to main acts of "ba.ra" in Genesis 1.
This new creation will be for not only Mount Zion itself, but also all of her (now) holy people.

{\vspace{1em}}

Her people get a very similar showing of Yahweh's divine presence in the "cloud by day"/"fire by night" that is a direct hyperlink to the installation of the Tabernacle where
Yahweh's glory that was leading them through the wilderness now was resting on the newly established tabernacle as His dwelling place.

\begin{quote}
\textit{
Then the cloud covered the tent of meeting, and the glory of the LORD filled the tabernacle.
And Moses was not able to enter the tent of meeting because the cloud settled on it, and the glory of the LORD filled the tabernacle. 
Throughout all their journeys, whenever the cloud was taken up from over the tabernacle, the people of Israel would set out. 
But if the cloud were not taken up, then they journeyed not till the day that it was taken up.
For the cloud of the LORD was on the tabernacle by day, and fire was in it by night, in the sight of all the house of Israel throughout all their journeys.
}
\\\\
\hfill --- Exodus 40:34-38
\end{quote}

{\vspace{1em}}

Now, in Isaiah's vision, this Tabernacle glory of Yahweh is available now to all of Zion and her people within it! God's dwelling place isn't just confined to a tabernacle or temple, it's now over all her assemblies in the form of a...wedding tent?

{\vspace{1em}}

The word "canopy" is a wedding tent (only ever used elsewhere in Psalm 19:5 and Joel 2:6). Instead of a regional temple, the whole mountain is a wedding tent where the ultimate union with the branch and His holy ones will occur.

\vspace{3em}
{\large\bfseries Safe from the Storm}
\vspace{1em}

In the final verse we see that the "wind of burning" and the "washing away" are still occuring in some way, except Zion's assemblies will all be safe from the storms around. It doesn't quite seem like the "final" image given in chapter 2 where there's no more war.

Just like in 1 Peter 1:5, God's people here are being guarded by His power for a salvation ready to be revealed in the last time.

This should give every reader in Jesus a hope and assurance of the "already" and the "not yet".

\begin{quote}
\textit{
    Isaiah believed that the final encampment of God's people would be in the new Zion. There, at last, their journey would end. But notice the democratization of the ancient ideal which takes place here. In the final encampment the glory of the Lord's presence fills the whole camp, and the protecting cloud, like a vast canopy or pavilion (cf. Exod. 40:34), covers the entire site and all who are assembled there (5). There will no longer be any need for the tabernacle or temple, for the glory of the Lord will be directly accessible to all. And those with whom God is present in this way will be perfectly secure for ever (6). This is no out-of-date dream, but one which Jesus prayed to be realized (John 17:24), and which the apostle John sets before us again at the climax of the Bible as the vision of our own future in God which should still inspire us and draw us on (Rev. 21:22-27). We, too, are pilgrims.
}
\\\\
\hfill --- Barry Webb, \textit{The Message of Isaiah}
\end{quote}

{\vspace{1em}}

\end{document}