\documentclass[11pt]{article}
\usepackage[margin=1in]{geometry}
\usepackage{../../styles/isaiah}
\usepackage{fontspec}
\usepackage{bidi}


\begin{document}


% Overview of Isaiah 1-5
\isaiahOverviewGrid{6}

% Overview of Isaiah 5
\begin{overview}{Isaiah 5:1-30 — Overview}

\overviewsection[0]{%
\textbf{Verses 1-7: Judgment on the Vineyard (A)}
\begin{itemize}
    \item A beloved's vineyard that \highlightgreen{planted} choice vines but \highlightred{yielded wild grapes}
    \item God \highlightpurple{looked for} \highlightaqua{justice} but found \highlightred{bloodshed and outcry}
\end{itemize}
}

\overviewsection[1]{%
\textbf{Verses 8-14: Two Woes and Two Therefores (B)}
\begin{itemize}
    \item \highlightorange{Woe} to greedy land-grabbers and indulgent party-goers
    \item \highlightbrown{Therefore} exile and death await the people
\end{itemize}}

\overviewsection[2]{%
\textbf{Verses 15-17: Divine Justice (C)}
\begin{itemize}
    \item Human pride humbled, God exalted in \highlightaqua{justice} and \highlightaqua{righteousness}
    \item Desolation where the rich once lived
\end{itemize}
}

\overviewsection[1]{%
\textbf{Verses 18-25: Four More Woes and Two Therefores (B')}
\begin{itemize}
    \item \highlightorange{Woe} to the provocative, morally confused, proud, and corrupt
    \item \highlightbrown{Therefore} God's anger burns against His people
\end{itemize}
}

\overviewsection[0]{%
\textbf{Verses 26-30: Judgment on the Land (A')}
\begin{itemize}
    \item God signals distant nations to come as His instrument of judgment
    \item \highlightgray{Darkness} and distress cover the land
\end{itemize}
}

\end{overview}

% Section A: Verses 1-7 - The Vineyard Parable
\begin{biblicaloutline}[Isaiah 5:1-7 (A)]
    
    \subsectionheader{The Song of the Vineyard (1-2)}

    \begin{versesection}{2em}
        \versenum{1} Let me sing for my beloved
        \poetryline my love song concerning his vineyard:
        My beloved had a vineyard
        \poetryline on a very fertile hill.
        
        \versenum{2} He dug it and cleared it of stones,
        \poetryline and \highlightgreen{planted} it with choice vines;
        he built a watchtower in the midst of it,
        \poetryline and hewed out a wine vat in it,
        and he \highlightpurple{looked for it to yield grapes},
        \poetryline but it \highlightred{yielded wild grapes}.
    \end{versesection}
    
    \subsectionheader{The Appeal to Judah (3-4)}
    
    \begin{versesection}{2em}
        \versenum{3} And now, O inhabitants of Jerusalem
        \poetryline and men of Judah,
        judge between me and my vineyard.
        
        \versenum{4} What more was there to do for my vineyard,
        \poetryline that I have not done in it?
        When I \highlightpurple{looked for it to yield grapes},
        \poetryline why did it \highlightred{yield wild grapes}?
    \end{versesection}
    
    \subsectionheader{The Judgment (5-6)}
    
    \begin{versesection}{2em}
        \versenum{5} And now I will tell you
        \poetryline what I will do to my vineyard.
        I will \highlightyellow{remove its hedge},
        \poetryline and it shall be devoured;
        I will break down its wall,
        \poetryline and it shall be trampled down.
        
        \versenum{6} I will make it a waste;
        \poetryline it shall not be pruned or hoed,
        \poetryline and briers and thorns shall grow up;
        I will also command the clouds
        \poetryline that they rain no rain upon it.
    \end{versesection}
    
    \subsectionheader{The Interpretation (7)}
    
    \begin{versesection}{2em}
        \versenum{7} For the vineyard of the LORD of hosts
        \poetryline is the house of Israel,
        and the men of Judah
        \poetryline are his pleasant \highlightgreen{planting};
        and he \highlightpurple{looked for} \highlightaqua{justice},
        \poetryline but \highlightred{behold, bloodshed};
        for \highlightaqua{righteousness},
        \poetryline but \highlightred{behold, an outcry}!
    \end{versesection}

\end{biblicaloutline}

\vspace{3em}
{\large\bfseries Jesus and the Vineyard Parable}
\vspace{1em}

This passage may sound familiar as it has the exact same intro to one of Jesus's parables about a vineyard as well in Matthew 21:33-46.

Both passages feature a carefully tended vineyard that fails to produce proper fruit for its owner.

\vspace{1em}

Taking the context here from Isaiah into account, reading Christ's parable becomes quite a bit more straightforward knowing what He's referencing.

Given Isaiah is one of the most often quoted books of the Old Testament in the New, It re-emphasizes our need to know the whole of the revealed scriptures. 

\vspace{3em}
{\large\bfseries Expectation vs Reality}
\vspace{1em}

\hebrewword{Wild Grapes}{בְּאֻשִׁים}{be.u.shim}{"stink fruit"}

In verses 2 and 4, we see at the end you have this expectation of what he looked for to yield grapes, but it instead yielded wild grapes or stink fruit. The next time you see this "he looked for" statement it's at the end of this little section in verse 7, when he is looking for justice and righteousness, but behold bloodshed and outcry. There's some wordplay going on here too:

\hebrewword{Justice}{מִשְׁפָּט}{mish.pat}{}
\hebrewword{Bloodshed}{מִשְׂפָּח}{mis.pach}{}

\hebrewword{Righteousness}{צְדָקָה}{tse.da.qah}{}
\hebrewword{Outcry}{צְעָקָה}{tse.a.qah}{}

Hebrew scholar Robert Alter in his translation of the Hebrew Bible gives the following to try and capture this:

\begin{quote}
\textit{"He hoped for justice,\\ and, look, jaundice,\\ for righteousness,\\ and, look, wretchedness."}\\\\
\hfill --- Robert Alter, \textit{The Hebrew Bible: A Translation with Commentary}
\end{quote}

\vspace{3em}
{\large\bfseries "What More Could God Do?"}
\vspace{1em}

The rhetorical question in verse 4 strikes at the heart of the tension between God's sovereignty and human responsibility and agency.  God gives a rhetorical statement here, implying that there is nothing else that he could do for the people themselves. 
\newpage

% Section B: Verses 8-14 - Woes Against Greed
\begin{chiasticoutline}[Isaiah 5:8-14 (B)]{.95em}{2em}

        \subsectionheader{First Woe: Greed (8-10)}
    \chiasticverse[A]{0}{
        
        \versenum{8} \highlightorange{Woe} to those who join house to house,

        \poetryline who add field to field,

        until there is no more room,

        \poetryline and you are made to dwell alone

        \poetryline in the midst of the land.
        
        \versenum{9} The LORD of hosts has sworn in my hearing:

        ``Surely many houses shall be desolate,

        \poetryline large and beautiful houses, without inhabitant.
        
        \versenum{10} For ten acres of vineyard shall yield but one bath,

        \poetryline and a homer of seed shall yield but an ephah.''
    }
   
    \subsectionheader{Second Woe: Indulgence (11-12)} 
        \chiasticverse[B]{1}{
        
        \versenum{11} \highlightorange{Woe} to those who rise early in the morning,

        \poetryline that they may run after strong drink,

        who tarry late into the evening

        \poetryline as wine inflames them!
        
        \versenum{12} They have lyre and harp,

        \poetryline tambourine and flute and wine at their feasts,

        but they do not regard the deeds of the LORD,

        \poetryline or \highlightblue{see the work of his hands}.
    }
     
    \subsectionheader{Two Therefores (13-14)}
    \chiasticverse[B']{1}{
        \versenum{13} \highlightbrown{Therefore} my people go into exile

        \poetryline for lack of knowledge;

        their honored men go hungry,

        \poetryline and their multitude is parched with thirst.
    }
    
    \chiasticverse[A']{0}{
        \versenum{14} \highlightbrown{Therefore} Sheol has enlarged its appetite

        \poetryline and opened its mouth beyond measure,

        and the nobility of Jerusalem and her multitude will go down,

        \poetryline her revelers and he who exults in her.
    }

\end{chiasticoutline}


\newpage
{\large\bfseries Whose Land?!}
\vspace{1em}

Leviticus 25:23 says, "The land shall not be sold in perpetuity, for the land is mine. For you are strangers and sojourners with me."

{\vspace{1em}}
The land ultimately belongs to God! And yet they want to take more and more at the expense of the needy – and what does it give them? Loneliness and isolation – kind of an ironic natural consequence of taking all the land for yourself.


\vspace{3em}
{\large\bfseries Ancient Measurements}

\begin{itemize}
    \item Acre: area a pair of oxen could plow in a day
    \item Bath: ~6 gallons (22 liters)
    \item Homer: 6 bushels (220–230 liters)
    \item Ephah: 0.6 bushels (22–23 liters)
\end{itemize}

\vspace{1em}

\vspace{3em}
{\large\bfseries Why are they being Exiled?}
\vspace{1em}

Due to a "lack of knowledge" Judah is being Exiled. From earlier sections like 1:12-15, we see that just because God's people know the right actions to take, their lack of knowledge of the true holy nature of Yahweh is what led them to their destruction.
\newpage
% Section C: Verses 15-17 - The Divine Response
\begin{biblicaloutline}[Isaiah 5:15-17 (C)]

    \begin{versesection}{2em}
        \versenum{15} Man is humbled, and each one is brought low,
        \poetryline and the eyes of the haughty are brought low.
        
        \versenum{16} But the LORD of hosts is exalted in \highlightaqua{justice},
        \poetryline and the Holy God shows himself holy in \highlightaqua{righteousness}.
        
        \versenum{17} Then shall the lambs graze as in their pasture,
        \poetryline and nomads shall eat among the ruins of the rich.
    \end{versesection}

\end{biblicaloutline}

% Section B': Verses 18-25 - More Woes
\begin{biblicaloutline}[Isaiah 5:18-25 (B')]
    
    \subsectionheader{Third Woe: Provocation (18-19)}

    \begin{versesection}{2em}
        \versenum{18} \highlightorange{Woe} to those who draw iniquity with cords of falsehood,
        \poetryline who draw sin as with cart ropes,
        
        \versenum{19} who say: ``Let him be quick,
        \poetryline let him \highlightblue{speed his work}
        \poetryline \highlightblue{that we may see it};
        let the counsel of the Holy One of Israel draw near,
        \poetryline and let it come, that we may know it!''
    \end{versesection}
    
    \subsectionheader{Fourth Woe: Moral Confusion (20)}
    
    \begin{versesection}{2em}
        \versenum{20} \highlightorange{Woe} to those who call evil good
        \poetryline and good evil,
        who put \highlightgray{darkness} for light
        \poetryline and light for \highlightgray{darkness},
        who put bitter for sweet
        \poetryline and sweet for bitter!
    \end{versesection}
    
    \subsectionheader{Fifth Woe: Pride (21)}
    
    \begin{versesection}{2em}
        \versenum{21} \highlightorange{Woe} to those who are wise in their own eyes,
        \poetryline and shrewd in their own sight!
    \end{versesection}
    
    \subsectionheader{Sixth Woe: Corruption (22-23)}
    
    \begin{versesection}{2em}
        \versenum{22} \highlightorange{Woe} to those who are heroes at drinking wine,
        \poetryline and valiant men in mixing strong drink,
        
        \versenum{23} who acquit the guilty for a bribe,
        \poetryline and \highlightyellow{deprive the innocent of his right}!
    \end{versesection}
    
    \subsectionheader{Two Therefores (24-25)}
    
    \begin{versesection}{2em}
        \versenum{24} \highlightbrown{Therefore}, as the tongue of fire devours the stubble,
        \poetryline and as dry grass sinks down in the flame,
        so their root will be as rottenness,
        \poetryline and their blossom go up like dust;
        for they have rejected the law of the LORD of hosts,
        \poetryline and have despised the word of the Holy One of Israel.
        
        \versenum{25} \highlightbrown{Therefore} the anger of the LORD was kindled against his people,
        \poetryline and he stretched out his hand against them and struck them,
        \poetryline and the mountains quaked;
        and their corpses were as refuse
        \poetryline in the midst of the streets.
        For all this his anger has not turned away,
        \poetryline and his hand is stretched out still.
    \end{versesection}

\end{biblicaloutline}

\vspace{2em}

\begin{biblecomparison}{Isaiah 5:23}
\translation{ESV}{who acquit the guilty for a bribe, and deprive the innocent of his right!}
\translation{NASB}{Who declare the wicked innocent for a bribe, And take away the rights of the ones who are in the right!}
\translation{NIV}{who acquit the guilty for a bribe, but deny justice to the innocent.}
\translation{KJV}{Which justify the wicked for reward, and take away the righteousness of the righteous from him!}
\translation{NET}{They pronounce the guilty innocent for a payoff, they ignore the just cause of the innocent.}
\end{biblecomparison}

\vspace{2em}

% Section A': Verses 26-30 - The Coming Invasion
\begin{biblicaloutline}[Isaiah 5:26-30 (A')]

    \begin{versesection}{2em}
        \versenum{26} He will raise a signal for nations far away,
        \poetryline and whistle for them from the ends of the earth;
        \poetryline and behold, quickly, speedily they come!
        
        \versenum{27} None is weary, none stumbles,
        \poetryline none slumbers or sleeps,
        not a waistband is loose,
        \poetryline not a sandal strap broken;
        
        \versenum{28} their arrows are sharp,
        \poetryline all their bows bent,
        their horses' hoofs seem like flint,
        \poetryline and their wheels like the whirlwind.
        
        \versenum{29} Their roaring is like a lion,
        \poetryline like young lions they roar;
        they growl and seize their prey;
        \poetryline they carry it off, and none can rescue.
        
        \versenum{30} They will growl over it on that day,
        \poetryline like the growling of the sea.
        And if one looks to the land,
        \poetryline behold, \highlightgray{darkness} and distress;
        and the light is \highlightgray{darkened} by its clouds.
    \end{versesection}

\end{biblicaloutline}


\newpage
{\large\bfseries Who's Really Destroying the Land?}
\vspace{1em}

When Yahweh "raises a signal" (something to watch for as we go), which is what causes the enemies to destroy the land.

While Israel may have often thought that Yahweh was absent in their terror and exile, He was not only there with them, but also the catalyst in the first place.

{\vspace{1em}}
Yahweh is in control from start to finish.

\vspace{3em}
{\large\bfseries Chaos abounds}
\vspace{1em}

Two of the main chaotic forces in the Hebrew Bible are the waters and the darkness. Genesis 1:2 starts off stating that these chaotic forces are what God overcomes in creating this world.

Here we see these chaotic forces at coming in for a prime de-creation. But if this is true, then what hope does Israel have to become the "Glorious Mountain" or the "Eden-like Wedding Tent"?

{\vspace{1em}}
What can make them new again?

\end{document}
