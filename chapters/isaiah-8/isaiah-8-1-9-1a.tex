\documentclass[11pt]{article}
\usepackage[margin=1in]{geometry}
\usepackage{../../styles/isaiah}
\usepackage{fontspec}
\usepackage{bidi}
\usepackage{../../styles/components/leftRightGrid}
\begin{document}
% Isaiah Context Grid
\leftRightGrid{3}{Isaiah 6-9:7}{
    \textbf{\large Isaiah 6}

    \textit{Isaiah's Call \& \\Commission}
}{
    \textbf{\large Isaiah 7}

    \textit{Immanuel}
}{
    \textbf{\large Isaiah 8}

    \textit{Maher-shalal-hash-baz}
}
{
    \textbf{\large Isaiah 9:1-7}

    \textit{The Prince of Peace}
}

% Overview of Isaiah 8:1-9:1a
\begin{overview}{Isaiah 8:1-9:1a — Overview}

\overviewsection[0]{%
\textbf{Verses 1-10: The Maher-Shalal-Hash-Baz Sign}
\begin{itemize}
    \item Prophet's son named "Swift to the Plunder"
    \item Assyria is coming to conquer Northern Kingdom
    \item God is with us (\highlightaqua{Immanuel}) despite judgment
\end{itemize}
}

\overviewsection[0]{%
\textbf{Verses 11-9:1a: A More Sure Rock}
\begin{itemize}
    \item Fear the Lord alone, not conspiracies
    \item God becomes sanctuary \textit{and} stumbling stone
    \item Hope in Yahweh and trust his teaching or face darkness
\end{itemize}
}

\end{overview}

\newpage

% Section 1: Verses 1-4
\begin{biblicaloutline}[Isaiah 8:1-4]

    \begin{versesection}{2em}
        \versenum{1} Then the LORD said to me, "Take a large tablet and write on it in \sectionwordfootnote{common characters}{"Writings of a man"}, 'Belonging to \sectionwordfootnote{Maher-shalal-hash-baz}{Quick to the Plunder. Swift to the Spoil}'" \versenum{2} And I will get reliable witnesses, Uriah the priest and Zechariah the son of Jeberechiah, to attest for me." \versenum{3} And I went to the prophetess, and she conceived and bore a son. Then the LORD said to me, "Call his name Maher-shalal-hash-baz; \versenum{4} \highlightgreen{for before the boy knows how to} cry 'My father' or 'My mother,' the wealth of Damascus and the spoil of Samaria will be carried away before the king of Assyria."
    \end{versesection}

\end{biblicaloutline}

\vspace{3em}
{\large\bfseries Uriah's Witness and the Two Child Signs}
\vspace{1em}

The mention of Uriah the priest as a witness (v. 2) creates a striking callback to 2 Kings 16:10-16, where Uriah helped King Ahaz construct a pagan altar, compromising his priestly calling. Now Uriah serves as witness to another child prophecy, creating a contrast between Ahaz's faithless witness in chapter 7 and this prophetic declaration.

\vspace{1em}
Both the Immanuel and Maher-Shalal-Hash-Baz prophecies share the identical phrase "before the boy knows how to" (v. 4, cf. 7:16), marking the timing of God's intervention. As this child grows to maturity, judgment will fall specifically on the Northern Kingdom, fulfilling God's promise to deliver Judah from her immediate enemies.

% Section 2: Verses 5-8
\begin{biblicaloutline}[Isaiah 8:5-8]

    \begin{versesection}{2em}
        \versenum{5} The LORD spoke to me again: \versenum{6} "Because this people has refused the waters of Shiloah that flow gently, and rejoice over Rezin and the son of Remaliah, \versenum{7} therefore, behold, the Lord is bringing up against them the waters of the River, mighty and many, the king of Assyria and all his glory. And it will rise over all its channels and go over all its banks, \versenum{8} and it will sweep on into Judah, it will overflow and pass on, reaching even to the neck, and its outstretched wings will fill the breadth of your land, O \highlightaqua{Immanuel}."
    \end{versesection}

\end{biblicaloutline}

\vspace{3em}
{\large\bfseries The Refused Waters of Shiloah}
\vspace{1em}

When verse 6 speaks of "this people" refusing "the waters of Shiloah that flow gently," it references the King's Pool in Jerusalem - the quiet, steady water source that sustained the holy city. This rejection symbolizes the Northern Kingdom's fundamental choice to worship apart from Jerusalem.

\vspace{1em}
This echoes the original kingdom split when Jeroboam set up golden calves at Dan and Bethel (1 Kings 12:28-29), declaring "it is too much for you to go up to Jerusalem." Rather than streaming to Zion as Isaiah envisioned (2:3), the Northern Kingdom established alternative worship centers, rejecting Jerusalem as God's chosen place. The following verses (7-8) reveal the devastating consequences: instead of Shiloah's gentle waters, they will face the overwhelming "waters of the River" - Assyria's destructive flood.

\vspace{3em}
{\large\bfseries Immanuel and the Land}
\vspace{1em}

The striking phrase "your land, O Immanuel" (v. 8) reveals the deep connection between the promised child and the land of Judah. Whether as future king and landowner or as one who shares in the judgment alongside his people, Immanuel is woven into the very fabric of the remnant's experience.

\vspace{1em}
This develops a central theme throughout Isaiah: the intertwining of the Messiah's identity with the remnant's identity. The Suffering Servant will bear the people's griefs (53:4), the Branch will emerge from Jesse's stump alongside the remnant (11:1, 10), and here Immanuel shares in the land's affliction. From Israel's calling as "my servant" (41:8-10) to the Servant's mission (49:1-6), Isaiah weaves together the destinies of the faithful remnant and their coming Messiah, showing that God's salvation comes through one who fully identifies with his people's plight.

% Section 3: Verses 9-10 (Poetry)
\begin{biblicaloutline}[Isaiah 8:9-10]

    \begin{versesection}{2em}
        \poetryline{\versenum{9} Be broken, you peoples, and be shattered;}
        \poetryline{give ear, all you far countries;}
        \poetryline{equip yourselves and be shattered;}
        \poetryline{equip yourselves and be shattered.}

        \poetryline{\versenum{10} Take counsel together, but it will come to nothing;}
        \poetryline{speak a word, but it will not stand,}
        \poetryline{for \highlightaqua{God is with us}.}
    \end{versesection}

\end{biblicaloutline}

\vspace{3em}
{\large\bfseries "God Is With Us"}
\vspace{1em}

The profound theological significance of "God is with us" (\hebrew{עמנואל}) cannot be overstated. As Oswalt explains, this phrase encapsulates the fundamental difference between biblical faith and all human religious attempts:

\begin{quote}
\textit{"It is hard to overstress the philosophical significance of God is with us. The nonbiblical approach is for an individual to seek to be with God—in fact to be united with God. This inevitably results in varying forms of pantheism or panentheism. If humanity is to attain unity with God it is impossible that God should transcend the psycho-physical world, for that world is finally our only means of access to him. But the biblical view exactly reverses the process. Transcendence is the given; it is nonnegotiable and irreducible. God is distinct from his world. This means that it is impossible for humanity to attain union with God by its devices. Instead God makes fellowship between us and him possible by entering our realm. Far from our trying to escape our finitude and mortality by making God identical to this world, God, who is part of this world, has entered into our finitude and mortality through Christ and thus brings us to fellowship with himself (John 3:13; Rom. 10:6; 2 Cor. 4:6; Col. 1:15-20)."}\\\\
\hfill --- John Oswalt, \textit{The Book of Isaiah, Chapters 1-39}
\end{quote}

\vspace{1em}
Here in chapter 8, even as judgment waters threaten to overwhelm, the declaration "God is with us" (v. 10) stands as the unshakeable foundation of hope. God does not abandon His people to face the flood alone - He enters their crisis, their land, their very existence.

% Section 4: Verses 11-15 with Chiastic Outline
\begin{biblicaloutline}[Isaiah 8:11-15]

    \begin{versesection}{2em}
        \versenum{11} For the LORD spoke thus to me with his strong hand upon me, and warned me not to walk in the way of this people, saying:
    \end{versesection}

    \begin{chiasticoutline}[]{.95em}{2em}
        
        \chiasticverse[A]{0}{
            \versenum{12} "Do not call conspiracy all that this people calls conspiracy, and do not fear what they fear, nor be in dread.
        }
        
        \chiasticverse[B]{1}{
            \versenum{13a} But the LORD of hosts, him you shall honor as \highlightyellow{holy}.
        }
        
        \chiasticverse[A']{0}{
            \versenum{13b} Let him be your fear, and let him be your dread. And he will become a sanctuary, 
        }
        {\vspace{2em}}
        \chiasticverse[A]{0}{
            \versenum{14a} but for both houses of Israel he will become a stone of \highlightbrown{stumbling} and a rock of offense,
        }

        \chiasticverse[B]{1}{
            \versenum{14b} a trap and a \highlightred{snare} to the inhabitants of Jerusalem.
        }

        \chiasticverse[A']{0}{
            \versenum{15a} And many shall \highlightbrown{stumble} on it. They shall fall and be broken;
        }

        \chiasticverse[B']{1}{
            \versenum{15b} they shall be \highlightred{snared} and taken."
        }

    \end{chiasticoutline}

\end{biblicaloutline}

\vspace{3em}
{\large\bfseries Peter's Use of Isaiah's Call to Holy Fear}
\vspace{1em}

The apostle Peter directly quotes this passage in 1 Peter 3:13-16, applying Isaiah's message to Christians facing persecution. Peter transforms Isaiah's warning about misplaced fear into encouragement for believers to maintain proper reverence for God alone:

\begin{quote}
\textit{"Now who is there to harm you if you are zealous for what is good? But even if you should suffer for righteousness' sake, you will be blessed. Have no fear of them, nor be troubled, but in your hearts honor Christ the Lord as holy, always being prepared to make a defense to anyone who asks you for a reason for the hope that is in you; yet do it with gentleness and respect, having a good conscience, so that, when you are slandered, those who revile your good behavior in Christ may be put to shame."}\\
\hfill --- 1 Peter 3:13-16
\end{quote}

\vspace{1em}

\begin{comparisontable}{Author}{Do Not Fear}{Honor as Holy}

\verserow{
Isaiah
}{
Assyrian threats and conspiracies
}{
Yahweh
}

\verserow{
Peter
}{
Persecutors of their Faith
}{
Christ the Lord
}


\end{comparisontable}

\vspace{1em}
Peter's quotation demonstrates that honoring Christ as holy is equivalent to honoring Yahweh as holy, affirming Christ's divine identity. The same God who called Isaiah's generation to trust Him alone in the face of Assyrian invasion now calls the church to honor Christ amid persecution, with the same promise: He will be either a sanctuary for those who trust or a stumbling stone for those who reject.

\vspace{3em}
{\large\bfseries The Real Issue: Power Politics or Divine Power?}
\vspace{1em}

Brevard Childs helps us understand the broader theological issue at stake in these verses:

\begin{quote}
\textit{"In my opinion, much of this debate has served as an unfortunate distraction from the main subject matter of the oracle. The divine warning does not turn on one form of political intrigue, which is never specified in the text itself. Rather, the warning is directed against all and everything that is surmised to be treason by the city's populace. As a result of such rumors the city is filled with great fear and foreboding of impending violence. In contrast, the prophet is called upon to direct his attention to the real source of power and dread: 'None but the LORD of hosts will you regard as holy.' He is the one to fear; he is the object of terror. In a word, the true issue at stake is again between two visions of reality. Does the future lie in the throes of power politics and clever human machinations, which evoke fear and uncertainty? Or does the future lie with God, the Holy One of Israel, who is the real power to be reckoned with?"}\\\\
\hfill --- Brevard Childs, \textit{Isaiah: A Commentary}
\end{quote}

\vspace{3em}
{\large\bfseries Jesus and the Rock Foundation}
\vspace{1em}

Jesus directly applies this imagery of the sanctuary rock in His Sermon on the Mount, concluding His teaching with a parable about two builders. The connection to Isaiah 8:14 becomes clear when we recognize that building on the rock means building on Christ's words—making Him both foundation and sanctuary:

\begin{quote}
\textit{"Everyone then who hears these words of mine and does them will be like a wise man who built his house on the rock. And the rain fell, and the floods came, and the winds blew and beat on that house, but it did not fall, because it had been founded on the rock. And everyone who hears these words of mine and does not do them will be like a foolish man who built his house on the sand. And the rain fell, and the floods came, and the winds blew and beat against that house, and it fell, and great was the fall of it."}\\
\hfill --- Matthew 7:24-27
\end{quote}

\vspace{1em}
The parallel is striking: Isaiah declares that the LORD will become either a sanctuary or a stone of stumbling (v. 14), while Jesus presents Himself as either a secure foundation or the cause of catastrophic collapse. Those who hear His words and obey find Him to be their sanctuary—their rock of refuge when the storms of judgment come. Those who hear and reject stumble over the very stone that could have saved them. Like the imagery of overwhelming flood waters in verse 8, Jesus speaks of rains, floods, and winds that test every foundation. Only those built upon Him as their rock will stand.

% Section 5: Verses 16-9:1a
\begin{biblicaloutline}[Isaiah 8:16-9:1a]
    \subsectionheader{Hope in Yahweh (16-18)}
    \begin{versesection}{2em}
        \versenum{16} Bind up the \highlightyellow{testimony}; seal the \highlightyellow{teaching} among my \sectionwordfootnote{disciples}{Either an official confirmation of authenticity – or – a result of failing to convince Ahaz}.\versenum{17} I will wait for the LORD, who is hiding his face from the house of Jacob, and I will hope in him. \versenum{18} Behold, I and the children whom the LORD has given me are signs and portents in Israel from the LORD of hosts, who dwells on Mount Zion. \versenum{19} And when they say to you, "Inquire of the \sectionwordfootnote{mediums}{Translated "ventriloquists" in the Septuagint} and the necromancers who chirp and mutter," should not a people inquire of their God? Should they inquire of the dead on behalf of the living?
    \end{versesection}

    \subsectionheader{Total Darkness (20-22; 9:1a)}
    \begin{versesection}{2em}
        \versenum{20} To the \highlightyellow{teaching} and to the \highlightyellow{testimony}! If they will not speak according to this word, it is because they have no dawn. \versenum{21} They will pass through the land, greatly distressed and hungry. And when they are hungry, they will be enraged and will speak contumely against their king and their God, and turn their faces upward. \versenum{22} And they will look to the earth, but behold, distress and darkness, the gloom of anguish. And they will be thrust into thick darkness.

        \versenum{9:1a} But there will be no gloom for her who was in anguish.
    \end{versesection}

\end{biblicaloutline}

\vspace{3em}
{\large\bfseries The Names: Isaiah's Family as Living Prophecy}
\vspace{1em}

When Isaiah declares "I and the children whom the LORD has given me are signs and portents in Israel" (v. 18), he reveals that his very family embodies the entire theological message of his book. Consider the names:

\vspace{1em}
\hebrewword{Isaiah}{יְשַׁעְיָהוּ}{ye.sha.ya.hu}{Yahweh is salvation}

\vspace{1em}
\hebrewword{Shear-Jashub}{שְׁאָר יָשׁוּב}{she.ar ya.shuv}{A remnant will return}

\vspace{1em}
\hebrewword{Maher-Shalal-Hash-Baz}{מַהֵר שָׁלָל חָשׁ בַּז}{ma.her sha.lal hash baz}{Swift is the booty, speedy is the prey}

\vspace{1em}
These three names together proclaim the book's central message: God saves (\textit{Isaiah}) through judgment (\textit{Maher-Shalal-Hash-Baz}), but a remnant will return (\textit{Shear-Jashub}). Every time Isaiah walked through Jerusalem with his sons, their very names preached the gospel—judgment is coming, yet hope remains for those who trust in Yahweh alone. The prophet's household becomes a living testimony that God's salvation comes not through political alliances or human schemes, but through divine judgment that purifies and preserves a faithful remnant.

\vspace{3em}
{\large\bfseries No Teaching, No Dawn}
\vspace{1em}

The declaration "if they will not speak according to this word, it is because they have no dawn" (v. 20) brings us back to the teaching and light wordplay that Isaiah established in chapter 2. There, the prophet envisioned the nations streaming to Jerusalem because "out of Zion shall go forth the law (\textit{torah}), and the word of the LORD from Jerusalem" (2:3), followed immediately by God's light breaking forth to establish justice. Now in chapter 8, we see the tragic inverse: rejecting the teaching (\textit{torah}) and testimony results in having "no dawn"—no light, only "distress and darkness, the gloom of anguish" and "thick darkness" (vv. 21-22).

\vspace{1em}
There is a direct line between rejecting the words of God and the destruction awaiting Israel. Without God's word, there is no light—only the thick darkness of judgment.

\end{document}