\documentclass[11pt]{article}
\usepackage[margin=1in]{geometry}
\usepackage{../../styles/isaiah}
\usepackage{fontspec}
\usepackage{bidi}

\begin{document}

% Section 1: Verses 1-4
\begin{biblicaloutline}[Isaiah 8:1-4]

    \begin{versesection}{2em}
        \versenum{1} Then the LORD said to me, "Take a large tablet and write on it in common characters, 'Belonging to Maher-shalal-hash-baz.'" \versenum{2} And I will get reliable witnesses, Uriah the priest and Zechariah the son of Jeberechiah, to attest for me." \versenum{3} And I went to the prophetess, and she conceived and bore a son. Then the LORD said to me, "Call his name Maher-shalal-hash-baz; \versenum{4} \highlightgreen{for before the boy knows how to} cry 'My father' or 'My mother,' the wealth of Damascus and the spoil of Samaria will be carried away before the king of Assyria."
    \end{versesection}

\end{biblicaloutline}

% Section 2: Verses 5-8
\begin{biblicaloutline}[Isaiah 8:5-8]

    \begin{versesection}{2em}
        \versenum{5} The LORD spoke to me again: \versenum{6} "Because this people has refused the waters of Shiloah that flow gently, and rejoice over Rezin and the son of Remaliah, \versenum{7} therefore, behold, the Lord is bringing up against them the waters of the River, mighty and many, the king of Assyria and all his glory. And it will rise over all its channels and go over all its banks, \versenum{8} and it will sweep on into Judah, it will overflow and pass on, reaching even to the neck, and its outstretched wings will fill the breadth of your land, O \highlightaqua{Immanuel}."
    \end{versesection}

\end{biblicaloutline}

% Section 3: Verses 9-10 (Poetry)
\begin{biblicaloutline}[Isaiah 8:9-10]

    \begin{versesection}{2em}
        \poetryline{\versenum{9} Be broken, you peoples, and be shattered;}
        \poetryline{give ear, all you far countries;}
        \poetryline{equip yourselves and be shattered;}
        \poetryline{equip yourselves and be shattered.}
        
        \poetryline{\versenum{10} Take counsel together, but it will come to nothing;}
        \poetryline{speak a word, but it will not stand,}
        \poetryline{for \highlightaqua{God is with us}.}
    \end{versesection}

\end{biblicaloutline}

% Section 4: Verses 11-15 with Chiastic Outline
\begin{biblicaloutline}[Isaiah 8:11-15]

    \begin{versesection}{2em}
        \versenum{11} For the LORD spoke thus to me with his strong hand upon me, and warned me not to walk in the way of this people, saying:
    \end{versesection}

    \begin{chiasticoutline}[]{.95em}{2em}
        
        \chiasticverse[A]{0}{
            \versenum{12} "Do not call conspiracy all that this people calls conspiracy, and do not fear what they fear, nor be in dread.
        }
        
        \chiasticverse[B]{1}{
            \versenum{13a} But the LORD of hosts, him you shall honor as \highlightyellow{holy}.
        }
        
        \chiasticverse[A']{0}{
            \versenum{13b} Let him be your fear, and let him be your dread.
        }

    \end{chiasticoutline}

    \begin{versesection}{2em}
        \versenum{14} And he will become a sanctuary, but for both houses of Israel he will become a stone of \highlightbrown{stumbling} and a rock of offense, a trap and a \highlightred{snare} to the inhabitants of Jerusalem. \versenum{15} And many shall \highlightbrown{stumble} on it. They shall fall and be broken; they shall be \highlightred{snared} and taken."
    \end{versesection}

\end{biblicaloutline}

% Section 5: Verses 16-9:1a
\begin{biblicaloutline}[Isaiah 8:16-9:1a]

    \begin{versesection}{2em}
        \versenum{16} Bind up the \highlightyellow{testimony}; seal the \highlightyellow{teaching} among my disciples. \versenum{17} I will wait for the LORD, who is hiding his face from the house of Jacob, and I will hope in him. \versenum{18} Behold, I and the children whom the LORD has given me are signs and portents in Israel from the LORD of hosts, who dwells on Mount Zion. \versenum{19} And when they say to you, "Inquire of the mediums and the necromancers who chirp and mutter," should not a people inquire of their God? Should they inquire of the dead on behalf of the living? \versenum{20} To the \highlightyellow{teaching} and to the \highlightyellow{testimony}! If they will not speak according to this word, it is because they have no dawn. \versenum{21} They will pass through the land, greatly distressed and hungry. And when they are hungry, they will be enraged and will speak contumely against their king and their God, and turn their faces upward. \versenum{22} And they will look to the earth, but behold, distress and darkness, the gloom of anguish. And they will be thrust into thick darkness.
        
        \versenum{9:1a} But there will be no gloom for her who was in anguish.
    \end{versesection}

\end{biblicaloutline}

\newpage
{\large\bfseries Uriah's Witness and the Two Child Signs}
\vspace{1em}

The mention of Uriah the priest as a witness (v. 2) creates a striking callback to 2 Kings 16:10-16, where Uriah helped King Ahaz construct a pagan altar, compromising his priestly calling. Now Uriah serves as witness to another child prophecy, creating a contrast between Ahaz's faithless witness in chapter 7 and this prophetic declaration.

\vspace{1em}
\hebrewword{Maher-Shalal-Hash-Baz}{מַהֵר שָׁלָל חָשׁ בַּז}{ma.her sha.lal hash baz}{Swift is the booty, speedy is the prey}

\vspace{1em}
This name directly contrasts with the Immanuel sign from chapter 7. While Immanuel ("God with us") offered comfort and divine presence, Maher-Shalal-Hash-Baz announces swift judgment. Yet both prophecies share the identical phrase "before the boy knows how to" (v. 4, cf. 7:16), marking the timing of God's intervention. As this child grows to maturity, judgment will fall specifically on the Northern Kingdom, fulfilling God's promise to deliver Judah from her immediate enemies.

\vspace{3em}
{\large\bfseries The Refused Waters of Shiloah}
\vspace{1em}

When verse 6 speaks of "this people" refusing "the waters of Shiloah that flow gently," it references the King's Pool in Jerusalem - the quiet, steady water source that sustained the holy city. This rejection symbolizes the Northern Kingdom's fundamental choice to worship apart from Jerusalem.

\vspace{1em}
This echoes the original kingdom split when Jeroboam set up golden calves at Dan and Bethel (1 Kings 12:28-29), declaring "it is too much for you to go up to Jerusalem." Rather than streaming to Zion as Isaiah envisioned (2:3), the Northern Kingdom established alternative worship centers, rejecting Jerusalem as God's chosen place. The following verses (7-8) reveal the devastating consequences: instead of Shiloah's gentle waters, they will face the overwhelming "waters of the River" - Assyria's destructive flood.

\vspace{3em}
{\large\bfseries Waters of Judgment}
\vspace{1em}

The imagery of Assyrian invasion as overwhelming waters "reaching even to the neck" (v. 8) taps into a profound biblical pattern of divine judgment through flood waters. From the primeval flood (Genesis 6-9) to the drowning of Pharaoh's army (Exodus 14-15), Scripture consistently portrays divine judgment as unstoppable waters that sweep away the rebellious.

\vspace{1em}
What makes this passage remarkable is its culmination: even as these waters of judgment fill the land, it remains "your land, O Immanuel" (v. 8). The child promised in chapter 7 is not absent from this crisis but intimately connected to it, suggesting his presence even in judgment.

\vspace{3em}
{\large\bfseries Immanuel and the Land}
\vspace{1em}

The striking phrase "your land, O Immanuel" (v. 8) reveals the deep connection between the promised child and the land of Judah. Whether as future king and landowner or as one who shares in the judgment alongside his people, Immanuel is woven into the very fabric of the remnant's experience.

\vspace{1em}
This develops a central theme throughout Isaiah: the intertwining of the Messiah's identity with the remnant's identity. The Suffering Servant will bear the people's griefs (53:4), the Branch will emerge from Jesse's stump alongside the remnant (11:1, 10), and here Immanuel shares in the land's affliction. From Israel's calling as "my servant" (41:8) to the Servant's mission (49:3-6), Isaiah weaves together the destinies of the faithful remnant and their coming Messiah, showing that God's salvation comes through one who fully identifies with his people's plight.

\vspace{3em}
{\large\bfseries "God Is With Us"}
\vspace{1em}

The profound theological significance of "God is with us" (\hebrew{עמנואל}) cannot be overstated. As Oswalt explains, this phrase encapsulates the fundamental difference between biblical faith and all human religious attempts:

\begin{quote}
\textit{"It is hard to overstress the philosophical significance of God is with us. The nonbiblical approach is for an individual to seek to be with God—in fact to be united with God. This inevitably results in varying forms of pantheism or panentheism. If humanity is to attain unity with God it is impossible that God should transcend the psycho-physical world, for that world is finally our only means of access to him. But the biblical view exactly reverses the process. Transcendence is the given; it is nonnegotiable and irreducible. God is distinct from his world. This means that it is impossible for humanity to attain union with God by its devices. Instead God makes fellowship between us and him possible by entering our realm. Far from our trying to escape our finitude and mortality by making God identical to this world, God, who is part of this world, has entered into our finitude and mortality through Christ and thus brings us to fellowship with himself (John 3:13; Rom. 10:6; 2 Cor. 4:6; Col. 1:15-20)."}\\\\
\hfill --- John Oswalt, \textit{The Book of Isaiah, Chapters 1-39}
\end{quote}

\vspace{1em}
Here in chapter 8, even as judgment waters threaten to overwhelm, the declaration "God is with us" (v. 10) stands as the unshakeable foundation of hope. God does not abandon His people to face the flood alone - He enters their crisis, their land, their very existence.

\end{document}