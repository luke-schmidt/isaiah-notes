%%%%%%%%%%%%%%%%%%%%%%%%%%%%%%%%%%%%%%%%%%%%%%%%%%%%%%%%%%%%%%%%%%%%%%%%%
%%
%% Isaiah 12 — Songs of Salvation
%%
%%%%%%%%%%%%%%%%%%%%%%%%%%%%%%%%%%%%%%%%%%%%%%%%%%%%%%%%%%%%%%%%%%%%%%%%%

\documentclass[11pt]{article}
\usepackage[margin=1in]{geometry}
\usepackage{../../styles/isaiah}

\begin{document}

%%%%%%%%%%%%%%%%%%%%%%%%%%%%%%%%%%%%%%%%%%%%%%%%%%%%%%%%%%%%%%%%%%%%%%%%%
%% Chiastic Overview
%%%%%%%%%%%%%%%%%%%%%%%%%%%%%%%%%%%%%%%%%%%%%%%%%%%%%%%%%%%%%%%%%%%%%%%%%

\isaiahChiasticOverview{7}

\newpage

%%%%%%%%%%%%%%%%%%%%%%%%%%%%%%%%%%%%%%%%%%%%%%%%%%%%%%%%%%%%%%%%%%%%%%%%%
%% Overview
%%%%%%%%%%%%%%%%%%%%%%%%%%%%%%%%%%%%%%%%%%%%%%%%%%%%%%%%%%%%%%%%%%%%%%%%%

\begin{overview}{Isaiah 12 — Songs of Salvation}
\overviewsection[0]{\textbf{Verses 1-2: Song 1}}
\overviewsection[1]{\textbf{Verse 3: Y'all will be Rescued}}
\overviewsection[0]{\textbf{Verses 4-5: Song 2}}
\end{overview}


%%%%%%%%%%%%%%%%%%%%%%%%%%%%%%%%%%%%%%%%%%%%%%%%%%%%%%%%%%%%%%%%%%%%%%%%%
%% Main Section
%%%%%%%%%%%%%%%%%%%%%%%%%%%%%%%%%%%%%%%%%%%%%%%%%%%%%%%%%%%%%%%%%%%%%%%%%

\begin{biblicaloutline}[Isaiah 12 — Songs of Salvation]



\begin{versesection}{2em}
\poetryline{\versenum{1} \sectionwordfootnote{\highlightyellow{You will say in that day:}}{Singular}}
\poetryline{\highlightyellow{"I will give thanks to you, O LORD,}}
\poetryline{\hspace{2em}for though you were angry with me,}
\poetryline{your anger turned away,}
\poetryline{\hspace{2em}that you might comfort me.}
\poetryline{\versenum{2} "Behold, God is my \highlightorange{salvation};}
\poetryline{\hspace{2em}I will trust, and will not be afraid;}
\poetryline{for the LORD GOD is my strength and my \highlightred{song},}
\poetryline{\hspace{2em}and he has become my \highlightorange{salvation}."}

\vspace{1em}

\poetryline{\versenum{3} With \highlightblue{joy} \sectionwordfootnote{you}{Plural} will draw water from the wells of \highlightorange{salvation}.}

\vspace{1em}

\poetryline{\versenum{4} And \sectionwordfootnote{\highlightyellow{you will say in that day:}}{Plural}}
\poetryline{\highlightyellow{"Give thanks to the LORD,}}
\poetryline{\hspace{2em}call upon his name,}
\poetryline{\highlightpurple{make known} his deeds among the peoples,}
\poetryline{\hspace{2em}proclaim that his name is exalted.}
\poetryline{\versenum{5} "\highlightred{Sing} praises to the LORD, for he has done gloriously;}
\poetryline{\hspace{2em}let this be \highlightpurple{made known} in all the earth.}
\poetryline{\highlightred{Shout, and sing} for \highlightblue{joy}, O inhabitant of Zion,}
\poetryline{\hspace{2em}for great in your midst is the Holy One of Israel."}
\end{versesection}

\end{biblicaloutline}

\vspace{2em}

%%%%%%%%%%%%%%%%%%%%%%%%%%%%%%%%%%%%%%%%%%%%%%%%%%%%%%%%%%%%%%%%%%%%%%%%%
%% Notes Section
%%%%%%%%%%%%%%%%%%%%%%%%%%%%%%%%%%%%%%%%%%%%%%%%%%%%%%%%%%%%%%%%%%%%%%%%%

{\large\bfseries Comparison with the Song of the Sea}
\vspace{1em}

This chapter mirrors the Song of the Sea from Exodus 15. After Moses led Israel through the Red Sea, they sang a song of thanksgiving and victory. After the Root of Jesse leads the remnant through the second exodus (chapter 11), they sing a similar song.

Compare the opening and closing of Isaiah 12 with Exodus 15:

\begin{quote}
\textit{Then Moses and the people of Israel sang this song to the LORD, saying,\\\\
"I will sing to the LORD, for he has triumphed gloriously;\\
\hspace{2em}the horse and his rider he has thrown into the sea.\\
The LORD is my strength and my song,\\
\hspace{2em}and he has become my salvation;\\
this is my God, and I will praise him,\\
\hspace{2em}my father's God, and I will exalt him.\\
The LORD is a man of war;\\
\hspace{2em}the LORD is his name..."}\\\\
\hfill --- Exodus 15:1-3
\end{quote}

\begin{quote}
\textit{Then Miriam the prophetess, the sister of Aaron, took a tambourine in her hand, and all the women went out after her with tambourines and dancing. And Miriam sang to them:\\\\
"Sing to the LORD, for he has triumphed gloriously;\\
\hspace{2em}the horse and his rider he has thrown into the sea."}\\\\
\hfill --- Exodus 15:20-21
\end{quote}

Note how both songs begin with thanksgiving to the LORD and end with singing and rejoicing. The pattern of deliverance leading to praise is central to Israel's identity.

\vspace{3em}

{\large\bfseries Salvation (3x) and Isaiah's Name}
\vspace{1em}

The word "salvation" appears three times in this short chapter (vv. 2, 2, 3), emphasizing the central theme of deliverance. This is significant because Isaiah's own name means "Yahweh is salvation" (\hebrew{יְשַׁעְיָהוּ}, \textit{Yeshayahu}).

The entire scroll of Isaiah is framed around this theme of God's salvation:
\begin{itemize}
\item Chapters 1-39: Judgment and the need for salvation
\item Chapters 40-55: The coming of salvation through the Servant
\item Chapters 56-66: The consummation of salvation in the new creation
\end{itemize}

Isaiah 12 serves as a hinge point in the first section, celebrating the salvation that God will accomplish through the Root of Jesse.

\newpage
{\large\bfseries How Can the Holy One Dwell in the Midst?}
\vspace{1em}

The final verse declares: "for great in your midst is the Holy One of Israel" (v. 6). But how can the Holy One of Israel be directly in the midst of sinful inhabitants?

We learned the answer in Chapter 6. When Isaiah encountered God's holiness in the temple, he cried out, "Woe is me! For I am lost; for I am a man of unclean lips, and I dwell in the midst of a people of unclean lips" (6:5).

The solution? Atonement. The seraph touched Isaiah's lips with a burning coal from the altar, saying, "Behold, this has touched your lips; your guilt is taken away, and your sin atoned for" (6:7).

Only through atonement can the Holy One dwell among unholy people. This points forward to the ultimate atonement that the Suffering Servant will accomplish (Isaiah 53), making it possible for God to dwell in the midst of His people forever.

\vspace{2em}

\end{document}
