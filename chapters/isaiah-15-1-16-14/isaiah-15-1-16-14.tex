\documentclass[11pt]{article}
\usepackage[margin=1in]{geometry}
\usepackage{../../styles/isaiah}
\usepackage{../../styles/components/threeBoxGrid}
\usepackage{../../styles/components/isaiahOraclesOverview}

\begin{document}

\newpage
\begin{center}
{\Huge\bfseries Isaiah 15:1-16:14}
\end{center}
\vspace{10pt}

% Oracle Overview for Chapters 13-23
\isaiahOraclesOverview{3}
\newpage
% Three Box Grid Overview
\threeBoxGrid{1}{Isaiah 15:1-16:14}{
    \textbf{\large Oracle against Moab}

    \textit{15:1-9}
}{
    \textbf{\large Moab running to Zion}

    \textit{16:1-5}
}{
    \textbf{\large Let Moab Wail}

    \textit{16:6-14}
}
{\vspace{4em}}
{\LARGE\bfseries Israel's History with Moab}
{\vspace{1em}}

Much of Israel's history with Moab up until this point is rocky at best. Here's a brief overview of key moments:

\vspace{1em}
\noindent\textbf{Genesis 19:30-38} — Origins

After Sodom and Gomorrah's destruction, Lot's daughters got him drunk and conceived children through him. The older daughter's son became Moab, making the Moabites distant relatives of Israel through Abraham's nephew, though marked by an incestuous and scandalous beginning.

\vspace{1em}
\noindent\textbf{Numbers 22-25} — Wilderness Period: Balaam and Baal of Peor

King Balak of Moab hired the prophet Balaam to curse Israel, but God transformed every curse into blessing (Numbers 22-24). However, Israelite men later engaged in sexual immorality with Moabite women and worshiped Baal of Peor, leading to a plague that killed 24,000 (Numbers 25).

\vspace{1em}
\noindent\textbf{Deuteronomy 23:3-6} — Exclusion from the Assembly

Because Moab did not meet Israel with bread and water during the exodus and hired Balaam to curse them, Moabites were prohibited from entering the assembly of the Lord for ten generations.

\vspace{1em}
\noindent\textbf{Ruth} — A Moabite Exception

Ruth the Moabite chose to follow her mother-in-law Naomi to Bethlehem, declaring "Your people shall be my people, and your God my God" (Ruth 1:16). Through her faithful devotion, she became the great-grandmother of King David.

\vspace{1em}
\noindent\textbf{Judges 3:12-30} — Oppression under King Eglon

The Moabite King Eglon oppressed Israel for eighteen years until the judge Ehud assassinated him, bringing eighty years of peace.

\vspace{1em}
\noindent\textbf{1 Samuel 14:47; 2 Samuel 8:2} — David's Conquest

King Saul fought against Moab. Later, David—despite his Moabite ancestry through Ruth—conquered Moab and made them a tributary nation, executing two-thirds of the prisoners and forcing the survivors to pay tribute.

% Chapter 15:1-5 - First Chiastic Structure
\begin{chiasticoutline}[Oracle against Moab - Isaiah 15:1-5]{.95em}{2em}


\chiasticverse[A]{0}{%
\versenum{1} An oracle concerning \highlightyellow{Moab}.\\
Because Ar of \highlightyellow{Moab} is laid waste in a night,\\
\poetryline \highlightyellow{Moab} is \sectionwordfootnote{undone}{'Da'mah' - only else used in chapter 6};\\
because Kir of \highlightyellow{Moab} is laid waste in a night,\\
\poetryline \highlightyellow{Moab} is undone.%
}


\chiasticverse[B]{1}{%
\versenum{2} He has gone up to the \sectionwordfootnote{temple}{"or 'bayith' - could be a city in Moab"}, and to \highlightgray{Dibon},\\
\poetryline to the high places to \highlightlightblue{weep};%
}


\chiasticverse[C]{2}{%
over Nebo and over Medeba\\
\poetryline \highlightyellow{Moab} \highlightdarkblue{wails}.\\
On every head is \sectionwordfootnote{baldness}{c.f. chapter 3};\\
\poetryline every beard is \sectionwordfootnote{shorn}{c.f. chapter 7}.\\
\versenum{3} In the streets they wear sackcloth;\\
\poetryline on the housetops and in the squares\\
\poetryline everyone \highlightdarkblue{wails} and \sectionwordfootnote{melts}{lit. 'goes down' - contrasted with housetops} in \highlightlightblue{tears}.%
}


\chiasticverse[D]{3}{%
\versenum{4} \highlightgray{Heshbon} and Elealeh \highlightblue{cry out};\\
\poetryline their voice is heard as far as Jahaz;\\
therefore the armed men of \highlightyellow{Moab} \highlightblue{cry aloud};\\
\poetryline his soul trembles.\\
\versenum{5} My heart \highlightblue{cries out} for \highlightyellow{Moab};\\
\poetryline her fugitives flee to Zoar,\\
\poetryline to Eglath-shelishiyah.%
}

\end{chiasticoutline}

{\vspace{4em}}
{\large\bfseries Fleeing to Zoar}
{\vspace{1em}}

The mention of fugitives fleeing to Zoar in verse 5 is deeply ironic given Moab's origins. Zoar was the small city where Lot fled when Sodom and Gomorrah were destroyed:

\begin{quote}
\textit{"As morning dawned, the angels urged Lot, saying, 'Up! Take your wife and your two daughters who are here, lest you be swept away in the punishment of the city.' But he lingered. So the men seized him and his wife and his two daughters by the hand, the LORD being merciful to him, and they brought him out and set him outside the city. And as they brought them out, one said, 'Escape for your life. Do not look back or stop anywhere in the valley. Escape to the hills, lest you be swept away.' And Lot said to them, 'Oh, no, my lords. Behold, your servant has found favor in your sight, and you have shown me great kindness in saving my life. But I cannot escape to the hills, lest the disaster overtake me and I die. Behold, this city is near enough to flee to, and it is a little one. Let me escape there—is it not a little one?—and my life will be saved!' He said to him, 'Behold, I grant you this favor also, that I will not overthrow the city of which you have spoken. Escape there quickly, for I can do nothing till you arrive there.' Therefore the name of the city was called Zoar. The sun had risen on the earth when Lot came to Zoar. Then the LORD rained on Sodom and Gomorrah sulfur and fire from the LORD out of heaven."} (Genesis 19:15-24)\\\\
\end{quote}

It was in the cave near Zoar where Lot's daughters conceived Moab and Ammon through incest with their father. Now Isaiah prophesies that the Moabites themselves must flee to Zoar—the very place of their shameful origin—because the destruction coming upon them mirrors the destruction of Sodom and Gomorrah that their ancestor escaped.

The name \textit{Eglath-shelishiyah} (meaning "third heifer" or possibly "heifer of three years") may also echo Judges 3, where Eglon, the obese king of Moab, oppressed Israel before being assassinated by Ehud—another unflattering association in Moab's history.
\newpage

% Chapter 15:6-9 - Second Chiastic Structure
\begin{chiasticoutline}[Oracle against Moab - Isaiah 15:6-9]{.95em}{2em}

\chiasticverse[D']{3}{%
For at the ascent of Luhith\\
\poetryline they go up \highlightlightblue{weeping};\\
on the road to Horonaim\\
\poetryline they raise a \highlightblue{cry} of destruction.\\
\versenum{7} Therefore the wealth they have gained\\
\poetryline and what they have laid up\\
they carry away\\
\poetryline over the Brook of the Willows.%
}

\chiasticverse[C']{2}{%
\versenum{8} For a \highlightblue{cry} has gone around\\
\poetryline the land of \highlightyellow{Moab};\\
her \highlightdarkblue{wailing} reaches to Eglaim;\\
\poetryline her \highlightdarkblue{wailing} reaches to Beer-elim.%
}

\chiasticverse[B']{1}{%
\versenum{9} For the waters of Dimon are full of blood;%
}

\chiasticverse[A']{0}{%
for I will bring upon Dimon even more,\\
\poetryline a lion for those of \highlightyellow{Moab} who escape,\\
\poetryline for the \highlightgreen{remnant} of the land.%
}

\end{chiasticoutline}

{\vspace{2em}}
{\large\bfseries Faith in Stuff}
{\vspace{1em}}

Verse 7 is emblematic of Moab's deeper problem - their misplaced faith in possessions and wealth. As they flee in panic, carrying "the wealth they have gained and what they have laid up," we see a people whose security was built on material prosperity rather than on God. This mirrors the indictment Isaiah brought against Judah earlier in his prophecy:

\begin{quote}
\textit{"Their land is filled with silver and gold, and there is no end to their treasures; their land is filled with horses, and there is no end to their chariots. Their land is filled with idols; they bow down to the work of their hands, to what their own fingers have made."} (Isaiah 2:7-8)
\end{quote}

Just as Judah trusted in their accumulated wealth, military might, and the work of their own hands, Moab's treasured possessions become a burden in the hour of judgment. What they thought would provide security must now be abandoned or carried away as they flee. The tragedy is not merely the loss of possessions, but the revelation that these things were never capable of providing the security and hope they placed in them.

\newpage

{\vspace{2em}}
{\large\bfseries Dibon/Dimon}
{\vspace{1em}}

Isaiah uses a wordplay in the oracle's conclusion. The city name shifts from \textbf{Dibon} (v. 2) to \textbf{Dimon} (v. 9), a deliberate alteration that evokes the Hebrew word for blood, \hebrewword{blood}{דָּם}{dam}{} Dibon becomes Dimon - the place of blood. The waters that should give life are instead "full of blood," transforming the city's very identity into a testimony of judgment.

{\vspace{4em}}
{\large\bfseries The Lion for Moab}
{\vspace{1em}}

The oracle's climax arrives with a declaration: \textit{"For the waters of Dibon are full of blood; for I will bring upon Dibon even more, a lion for those of Moab who escape, for the remnant of the land"} (15:9).

Lions and Moab share an interesting history:

\begin{quote}
\textit{"And Benaiah the son of Jehoiada was a valiant man of Kabzeel, a doer of great deeds. He struck down two ariels of Moab. He also went down and struck down a lion in a pit on a day when snow had fallen."}\\
\hfill --- 2 Samuel 23:20
\end{quote}

Benaiah, one of David's mighty men, struck down two "ariels of Moab" - the term "ariel" (literally "lion of God") likely refers to Moabite warriors or champions. He also killed an actual lion in a pit on a snowy day. 

Now Isaiah prophesies that Yahweh himself will send lions after Moab's escapees - those who flee the Assyrian invasion will face an even more terrifying predator. The remnant that survives human armies will not escape divine judgment.

\newpage
% Update Three Box Grid to highlight box 2
\threeBoxGrid{2}{Isaiah 15:1-16:14}{
    \textbf{\large Oracle against Moab}

    \textit{15:1-9}
}{
    \textbf{\large Moab running to Zion}

    \textit{16:1-5}
}{
    \textbf{\large Let Moab Wail}

    \textit{16:6-14}
}

% Chapter 16:1-5 - Moab Running to Zion
\begin{biblicaloutline}[Moab Running to Zion - Isaiah 16:1-5]

\begin{versesection}{2em}
\versenum{1} Send the lamb to the ruler of the land,
\poetryline from Sela, by way of the desert,
\poetryline to the mount of the daughter of Zion.
\versenum{2} Like fleeing birds,
\poetryline like a scattered nest,
so are the daughters of \highlightyellow{Moab}
\poetryline at the fords of the Arnon.

\versenum{3} "Give counsel;
\poetryline grant justice;
make your shade like night
\poetryline at the height of noon;
shelter the outcasts;
\poetryline do not reveal the fugitive;
\versenum{4} let the outcasts of \highlightyellow{Moab}
\poetryline \sectionwordfootnote{sojourn among you}{or 'let my outcasts dwell among you, Moab'};
be a shelter to them
\poetryline from the destroyer.
When the oppressor is no more,
\poetryline and destruction has ceased,
and he who tramples underfoot has vanished from the land,
\versenum{5} then a throne will be established in steadfast love,
\poetryline and on it will sit in faithfulness
\poetryline in the tent of David
one who judges and seeks \highlightaqua{justice}
\poetryline and is swift to do \highlightaqua{righteousness}."
\end{versesection}

\end{biblicaloutline}

{\vspace{2em}}
{\large\bfseries Send the Lamb}
{\vspace{1em}}

The command to "send the lamb to the ruler of the land" (16:1) has historical precedent. Moab had a long history of sheep breeding and sending tribute to Israelite kings.

\begin{quote}
\textit{"Now Mesha king of Moab was a sheep breeder, and he had to deliver to the king of Israel 100,000 lambs and the wool of 100,000 rams."} (2 Kings 3:4)
\end{quote}

This tribute relationship extended back even further to David's reign:

\begin{quote}
\textit{"And he defeated Moab and he measured them with a line, making them lie down on the ground. Two lines he measured to be put to death, and one full line to be kept alive. And the Moabites became servants to David and brought tribute."} (2 Samuel 8:2)
\end{quote}

The call to send lambs to the Davidic ruler is thus a plea to restore this ancient relationship of tribute and protection—to acknowledge the Davidic king and seek refuge under his authority.

{\vspace{4em}}
{\large\bfseries Running to Zion}
{\vspace{1em}}

This passage reveals a remarkable fulfillment of Isaiah's earlier vision of the nations streaming to Zion:

\begin{quote}
\textit{"It shall come to pass in the latter days that the mountain of the house of the LORD shall be established as the highest of the mountains, and shall be lifted up above the hills; and all the nations shall flow to it, and many peoples shall come, and say: 'Come, let us go up to the mountain of the LORD, to the house of the God of Jacob, that he may teach us his ways and that we may walk in his paths.' For out of Zion shall go forth the law, and the word of the LORD from Jerusalem."} (Isaiah 2:2-3)
\end{quote}

Moab's flight to Zion is not merely political refuge—it's an oracle of the nations coming to worship the God of Israel. The imagery of "fleeing birds, like a scattered nest" (16:2) portrays desperate Moabite women seeking shelter at the borders of Judah. Their plea for protection is a petition to be covered by Zion's shadow "at the height of noon" (16:3), to dwell under the Davidic throne "established in steadfast love" and ruled in "faithfulness" (16:5).

The last time a Moabite fugitive fled to Zion seeking refuge, it was Ruth—the widow who chose Naomi's God over her homeland's gods. Her famous declaration—"Your people shall be my people, and your God my God" (Ruth 1:16)—led her to Bethlehem, to Boaz, and ultimately into the lineage of David and Christ himself. Ruth's faith transformed her from outsider to matriarch, from Moabite to great-grandmother of Israel's greatest king.

Now Isaiah prophesies that Moab as a nation must make Ruth's choice. Will they flee to the "tent of David" (16:5) and acknowledge the Davidic king who judges with justice and righteousness? Or will their pride prevent them from seeking refuge in the very place their foremother found redemption?

The nations flowing to Zion is not a future hope alone—it begins with Moab's desperate flight to Jerusalem, seeking protection from "the destroyer" under the shadow of the mountain of the LORD.

{\vspace{4em}}
{\large\bfseries Throne in Steadfast Love and Faithfulness}
{\vspace{1em}}

The promise of a throne "established in steadfast love" with a ruler who sits "in faithfulness" (16:5) echoes the most frequently quoted verse in the entire Hebrew Bible—God's self-revelation to Moses:

\begin{quote}
\textit{"The LORD passed before him and proclaimed, 'The LORD, the LORD, a God merciful and gracious, slow to anger, and abounding in steadfast love and faithfulness.'"}\\
\hfill --- Exodus 34:6
\end{quote}

Isaiah's vision of the Davidic throne reveals that this divine nature will be embodied in a human king. The one who sits on David's throne will not merely represent God's justice and righteousness abstractly—he will \textit{be} like God himself, ruling with the very character attributes that define Yahweh. Where other kings rule through power, intimidation, or political expediency, this coming king will establish his throne through \textit{hesed} (steadfast love) and \textit{emet} (faithfulness).

For the fleeing Moabites seeking refuge, this is crucial. They don't need a king who will exploit their vulnerability or demand tribute without protection. They need what Ruth found in Boaz—a redeemer whose kindness flows from his character, not from political calculation. The throne "established in steadfast love" offers genuine refuge because its foundation is the same character that defines the God who abounds in mercy and grace.

This is the throne to which the nations will stream. This is the king for whom creation groans.

\newpage

% Update Three Box Grid to highlight box 3
\threeBoxGrid{3}{Isaiah 15:1-16:14}{
    \textbf{\large Oracle against Moab}

    \textit{15:1-9}
}{
    \textbf{\large Moab running to Zion}

    \textit{16:1-5}
}{
    \textbf{\large Let Moab Wail}

    \textit{16:6-14}
}

% Chapter 16:6-14 - Let Moab Wail (using overview component)
\begin{overview}{Let Moab Wail - Isaiah 16:6-14}

\overviewsection[0]{%
\textbf{16:6 — Pride}
\begin{itemize}
    \item Pride, arrogance, insolence, boasts - empty
\end{itemize}
}

\overviewsection[1]{%
\textbf{16:7-8 — \highlightorange{Therefore} \#1: Moab Wail for Kir}
\begin{itemize}
    \item \highlightyellow{Moab} \highlightdarkblue{wails} for \highlightgray{Kir-hareseth}
\end{itemize}
}

\overviewsection[1]{%
\textbf{16:9-10 — \highlightorange{Therefore} \#2: I Weep for Moab}
\begin{itemize}
    \item I drench you with my \highlightlightblue{tears}, O \highlightgray{Heshbon} and Elealeh
\end{itemize}
}

\overviewsection[1]{%
\textbf{16:11 — \highlightorange{Therefore} \#3: I Moan for Kir}
\begin{itemize}
    \item My inner parts moan like a lyre for \highlightyellow{Moab}
    \item My inmost self for \highlightgray{Kir-hareseth}
\end{itemize}
}

\overviewsection[0]{%
\textbf{16:12-14 — Consequence of Pride}
\begin{itemize}
    \item Three years: glory brought into contempt, \highlightgreen{remnant} small and feeble
\end{itemize}
}

\end{overview}

% Full text of 16:6-10
\begin{biblicaloutline}[Let Moab Wail - Isaiah 16:6-10]

\subsectionheader{Pride of Moab}

\begin{versesection}{2em}
\versenum{6} We have heard of the pride of \highlightyellow{Moab}—
\poetryline how proud he is!—
of his arrogance, his pride, and his insolence;
\poetryline in his idle boasting he is not right.
\end{versesection}

\subsectionheader{Therefore \#1: Moab Wails}

\begin{versesection}{2em}
\versenum{7} \highlightorange{Therefore} let \highlightyellow{Moab} \highlightdarkblue{wail} for \highlightyellow{Moab},
\poetryline let everyone \highlightdarkblue{wail}.
Mourn, utterly stricken,
\poetryline for the \sectionwordfootnote{raisin cakes}{could be cultic significance (Hos 3:1) or just an extrapolation of v8} of \highlightgray{Kir-hareseth}.

\versenum{8} For the fields of \highlightgray{Heshbon} languish,
\poetryline and the \highlightpurple{vine of Sibmah};
the lords of the nations
\poetryline have struck down its choice clusters
that reached to \highlightgray{Jazer}
\poetryline and strayed to the desert;
its shoots spread abroad
\poetryline and passed over the sea.
\end{versesection}

\subsectionheader{Therefore \#2: I Weep}

\begin{versesection}{2em}
\versenum{9} \highlightorange{Therefore} I \highlightlightblue{weep} with the \highlightlightblue{weeping} of \highlightgray{Jazer}
\poetryline for the \highlightpurple{vine of Sibmah};
I drench you with my \highlightlightblue{tears},
\poetryline O \highlightgray{Heshbon} and Elealeh;
for over your summer fruit and your harvest
\poetryline the \highlightaqua{shout} has ceased.
\versenum{10} And joy and gladness are taken away
\poetryline from the fruitful field,
and in the vineyards no songs are sung,
\poetryline no \highlightaqua{cheers} are raised;
no treader treads out wine in the presses;
\poetryline I have put an end to the \highlightaqua{shouting}.
\end{versesection}

\end{biblicaloutline}

% Full text of 16:11-14
\begin{biblicaloutline}[Let Moab Wail - Isaiah 16:11-14 (cont.)]

\subsectionheader{Therefore \#3: I Moan}

\begin{versesection}{2em}
\versenum{11} \highlightorange{Therefore} my inner parts moan like a lyre for \highlightyellow{Moab},
\poetryline and my inmost self for \highlightgray{Kir-hareseth}.
\end{versesection}

\subsectionheader{Consequence of Pride}

\begin{versesection}{2em}
\versenum{12} And when \highlightyellow{Moab} presents himself, when he wearies himself on the \highlightsilver{high place}, when he comes to his sanctuary to pray, he will not prevail.

\versenum{13} This is the word that the LORD spoke concerning \highlightyellow{Moab} in the \sectionwordfootnote{past}{a little glimpse into the making of Isaiah}.
\versenum{14} But now the LORD has spoken, saying, "In three years, like the years of a hired worker, the glory of \highlightyellow{Moab} will be brought into contempt, in spite of all his \sectionwordfootnote{great multitude}{c.f. 15:7}, and those who \highlightgreen{remain} will be very few and feeble."
\end{versesection}

\end{biblicaloutline}

{\vspace{4em}}
{\large\bfseries The Prophet's Heart: Isaiah's Lament}
{\vspace{1em}}

Three times Isaiah interrupts the oracle with his own emotional response, and each time the intensity deepens:

\begin{itemize}
    \item \textbf{15:5} — "My heart \highlightblue{cries out} for Moab"
    \item \textbf{16:9} — "I \highlightlightblue{weep}...I drench you with my \highlightlightblue{tears}"
    \item \textbf{16:11} — "My inner parts moan like a lyre for Moab, my inmost self"
\end{itemize}

Isaiah prophesies Moab's destruction—and then weeps over it. He announces God's judgment—and his heart breaks with it. The prophet doesn't stand apart from the suffering he proclaims; he enters into it with visceral, physical anguish that moves from his heart, through his tears, into his very inmost being. This is how God's messengers deliver hard words: not with satisfaction, but with tears that demonstrate judgment flows from love, not indifference.

\end{document}
