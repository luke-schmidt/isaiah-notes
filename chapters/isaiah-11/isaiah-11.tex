%%%%%%%%%%%%%%%%%%%%%%%%%%%%%%%%%%%%%%%%%%%%%%%%%%%%%%%%%%%%%%%%%%%%%%%%%
%%
%% Isaiah 11 — The Root of Jesse and Second Exodus
%%
%%%%%%%%%%%%%%%%%%%%%%%%%%%%%%%%%%%%%%%%%%%%%%%%%%%%%%%%%%%%%%%%%%%%%%%%%

\documentclass[11pt]{article}
\usepackage[margin=1in]{geometry}
\usepackage{../../styles/isaiah}
\usepackage{graphicx}
\usepackage{../../styles/components/threeBoxGrid}

\begin{document}

%%%%%%%%%%%%%%%%%%%%%%%%%%%%%%%%%%%%%%%%%%%%%%%%%%%%%%%%%%%%%%%%%%%%%%%%%
%% Chiastic Overview
%%%%%%%%%%%%%%%%%%%%%%%%%%%%%%%%%%%%%%%%%%%%%%%%%%%%%%%%%%%%%%%%%%%%%%%%%

\isaiahChiasticOverview{6}

\newpage

%%%%%%%%%%%%%%%%%%%%%%%%%%%%%%%%%%%%%%%%%%%%%%%%%%%%%%%%%%%%%%%%%%%%%%%%%
%% Overview
%%%%%%%%%%%%%%%%%%%%%%%%%%%%%%%%%%%%%%%%%%%%%%%%%%%%%%%%%%%%%%%%%%%%%%%%%

\begin{overview}{Isaiah 11 — The Root of Jesse/Second Exodus}
\overviewsection[0]{\textbf{Verses 1-10: Who is the Shoot of Jesse? and What is His Reign Like?}}
\overviewsection[0]{\textbf{Verses 11-16: How Will He Accomplish This Reign?}}
\end{overview}

\vspace{2em}

%%%%%%%%%%%%%%%%%%%%%%%%%%%%%%%%%%%%%%%%%%%%%%%%%%%%%%%%%%%%%%%%%%%%%%%%%
%% Section 1: Who is the Shoot of Jesse? (vv.1-5)
%%%%%%%%%%%%%%%%%%%%%%%%%%%%%%%%%%%%%%%%%%%%%%%%%%%%%%%%%%%%%%%%%%%%%%%%%


\begin{biblicaloutline}[Isaiah 11:1-5 — Who is the Shoot of Jesse?]

\begin{versesection}{2em}
\poetryline{\versenum{1} There shall come forth a shoot from the stump of \highlightpurple{Jesse},}
\poetryline{\hspace{2em}and a branch from his \highlightpurple{roots} shall bear fruit.}
\poetryline{\versenum{2} And the \highlightorange{Spirit} of the LORD shall \sectionwordfootnote{rest}{c.f. Luke 3:22} upon him,}
\poetryline{\hspace{2em}the \highlightorange{Spirit} of wisdom and understanding,}
\poetryline{\hspace{2em}the \highlightorange{Spirit} of counsel and \sectionwordfootnote{might}{Callback to 9:1-7 with Mighty Counselor},}
\poetryline{\hspace{2em}the \highlightorange{Spirit} of knowledge and the fear of the LORD.}
\poetryline{\versenum{3} And his \sectionwordfootnote{delight}{lit. "smelling" \hebrew{ריח} [riach]} shall be in the fear of the LORD.}
\poetryline{He shall not \highlightaqua{judge} by what his eyes see,}
\poetryline{\hspace{2em}or decide disputes by what his ears hear,}
\poetryline{\versenum{4} but with \highlightaqua{righteousness} he shall \highlightaqua{judge} the poor,}
\poetryline{\hspace{2em}and decide with equity for the \sectionwordfootnote{meek}{the afflicted} of the earth;}
\poetryline{and he shall strike the earth with the rod of his mouth,}
\poetryline{\hspace{2em}and with the \highlightorange{breath} of his lips he shall kill the wicked.}
\poetryline{\versenum{5} \highlightaqua{Righteousness} shall be the belt of his waist,}
\poetryline{\hspace{2em}and faithfulness the belt of his loins.}
\end{versesection}

\end{biblicaloutline}

\newpage

{\large\bfseries What Comes from the Stump?}
\vspace{1em}

Compare the end of chapter 6 with the "seed" coming from the stump and this passage.
\begin{quote}
\textit{And though a tenth remain in it, it will be burned again, like a terebinth or an oak, whose stump remains when it is felled. The holy seed is its stump.} (Isaiah 6:13)
\end{quote}

Like we talked about in chapter 7 when comparing the signs for Ahaz and Hezekiah, the Messiah is very closely intertwined with His followers — they are inseparable in God's plan of redemption.

\vspace{3em}

{\large\bfseries From Jesse, not David}
\vspace{1em}

Israel doesn't need another king from David — they all failed.
\vspace{1em}

What Israel needs is a brand new David. Contrast these characteristics of the Root of Jesse with what Ahaz has looked like in the previous passages:

\begin{center}
\begin{tabular}{|l|l|}
\hline
\textbf{Ahaz} & \textbf{The Root of Jesse (Messiah)} \\
\hline
Misplaced fear & Right fear (fear of the LORD) \\
\hline
No wisdom & \highlightorange{Spirit} of wisdom and understanding \\
\hline
Led to exile & Leads back from the exile \\
\hline
\end{tabular}
\end{center}

\vspace{2em}

{\large\bfseries Weapon of Destruction}
\vspace{1em}

His Word is what's doing the destroying here (v.4). It's "decreating" as well as creates (Genesis 1). The same divine word that spoke creation into existence now speaks judgment and restoration.

John picks this image up in Revelation 19:15, describing Jesus' return:
\begin{quote}
\textit{"From his mouth comes a sharp sword with which to strike down the nations, and he will rule them with a rod of iron."}\\\\
\hfill --- Revelation 19:15
\end{quote}

\vspace{2em}

%%%%%%%%%%%%%%%%%%%%%%%%%%%%%%%%%%%%%%%%%%%%%%%%%%%%%%%%%%%%%%%%%%%%%%%%%
%% Section 2: What is His Reign Like? (vv.6-9)
%%%%%%%%%%%%%%%%%%%%%%%%%%%%%%%%%%%%%%%%%%%%%%%%%%%%%%%%%%%%%%%%%%%%%%%%%

\begin{biblicaloutline}[Isaiah 11:6-9 — What is His Reign Like?]

\begin{versesection}{2em}
\poetryline{\versenum{6} The wolf shall dwell with the lamb,}
\poetryline{\hspace{2em}and the leopard shall lie down with the young goat,}
\poetryline{and the calf and the lion and the \sectionwordfootnote{fattened calf together}{even a tempting, fat cow won't do};}
\poetryline{\hspace{2em}and a little child shall lead them.}
\poetryline{\versenum{7} The cow and the bear shall graze;}
\poetryline{\hspace{2em}their young shall lie down together;}
\poetryline{\hspace{2em}and the lion shall eat straw like the ox.}
\poetryline{\versenum{8} The nursing child shall play over the hole of the cobra,}
\poetryline{\hspace{2em}and the weaned child shall put his hand on the adder's \sectionwordfootnote{den}{Genesis 3:15}.}
\poetryline{\versenum{9} They shall not hurt or destroy}
\poetryline{\hspace{2em}in all my holy mountain;}
\poetryline{for the earth shall be full of the knowledge of the LORD}
\poetryline{\hspace{2em}as the waters cover the sea.}
\end{versesection}

\end{biblicaloutline}

\vspace{2em}

{\large\bfseries Knowing Yahweh Leads to Peace}
\vspace{1em}

Like chapter 2, what does learning the instruction of the LORD and learning Who He is lead to? Not continuing wars or conquests but peace. The knowledge of God transforms creation itself, reversing the curse and restoring the harmony of Eden.

\vspace{2em}

%%%%%%%%%%%%%%%%%%%%%%%%%%%%%%%%%%%%%%%%%%%%%%%%%%%%%%%%%%%%%%%%%%%%%%%%%
%% Section 3: Jesus as the Banner (vv.10-12)
%%%%%%%%%%%%%%%%%%%%%%%%%%%%%%%%%%%%%%%%%%%%%%%%%%%%%%%%%%%%%%%%%%%%%%%%%

\begin{biblicaloutline}[Isaiah 11:10-12 — Jesus as the Banner/Signal]

\begin{versesection}{2em}
\poetryline{\versenum{10} \highlightgreen{In that day} the \sectionwordfootnote{\highlightpurple{root of Jesse}}{In v.1, the Messiah is a "shoot from the stump of Jesse" and a "branch from his roots." By v.10, He has become simply "the root of Jesse" — no longer just a descendant, but the very source itself. The branch has become the root.}, who shall stand as a \highlightyellow{signal for the peoples}—}
\poetryline{\hspace{2em}of him shall the nations inquire,}
\poetryline{\hspace{2em}and his resting place shall be glorious.}
\poetryline{\versenum{11} \highlightgreen{In that day} the Lord will extend his hand yet a \sectionwordfootnote{second time}{When was the first? Keep reading} to recover \highlightblue{the remnant that remains of his people},}
\poetryline{\hspace{2em}from \highlightgray{Assyria}, from \highlightgray{Egypt}, from Pathros, from Cush, from Elam, from Shinar, from Hamath, and from the}
\poetryline{\hspace{2em}coastlands of the sea.}
\poetryline{\versenum{12} He will raise a \highlightyellow{signal for the nations}}
\poetryline{\hspace{2em}and will assemble the banished of Israel,}
\poetryline{and gather the dispersed of Judah}
\poetryline{\hspace{2em}from the four corners of the earth.}
\end{versesection}

\end{biblicaloutline}

\vspace{2em}

{\large\bfseries Jesus as the Banner}
\vspace{1em}

In Exodus 17:8-15, when Israel fought the Amalekites, Moses raised his staff as a banner of victory. When raised on high hilltops, a banner was used to rally the troops and signal where God's presence was leading.

John 12:32 shows Jesus has the same understanding of his death and resurrection:

\begin{quote}
\textit{"And I, when I am lifted up from the earth, will draw all people to myself."}\\\\
\hfill --- John 12:32
\end{quote}

This Banner answers the question posed in chapter 2: What's causing the nations to "river up" to the mountain of Yahweh?

Also, this banner in v12 will reverse the exact "signal/banner" used in 5:26, where God raised a signal for distant nations to come as instruments of judgment against Israel. Now, the signal gathers the scattered remnant \textit{home}.

Compare v11 with:

\begin{quote}
\textit{"He will raise a signal for nations far away, and whistle for them from the ends of the earth; and behold, quickly, speedily they come! None is weary, none stumbles, none slumbers or sleeps, not a waistband is loose, not a sandal strap broken; their arrows are sharp, all their bows bent, their horses' hoofs seem like flint, and their wheels like the whirlwind. Their roaring is like a lion, like young lions they roar; they growl and seize their prey; they carry it off, and none can rescue."}\\\\
\hfill --- Isaiah 5:26-29
\end{quote}

\vspace{2em}

%%%%%%%%%%%%%%%%%%%%%%%%%%%%%%%%%%%%%%%%%%%%%%%%%%%%%%%%%%%%%%%%%%%%%%%%%
%% Section 4: How Will He Accomplish This? (vv.13-16)
%%%%%%%%%%%%%%%%%%%%%%%%%%%%%%%%%%%%%%%%%%%%%%%%%%%%%%%%%%%%%%%%%%%%%%%%%

\begin{biblicaloutline}[Isaiah 11:13-16 — How Will He Accomplish This Reign?]

\begin{versesection}{2em}
\poetryline{\versenum{13} The jealousy of Ephraim shall depart,}
\poetryline{\hspace{2em}and those who harass Judah shall be cut off;}
\poetryline{Ephraim shall not be jealous of Judah,}
\poetryline{\hspace{2em}and Judah shall not harass Ephraim.}
\poetryline{\versenum{14} But they shall swoop down on the shoulder of the \sectionwordfootnote{Philistines}{Figurative language — Philistines didn't have a distinct national identity in Isaiah's day} in the west,}
\poetryline{\hspace{2em}and together they shall plunder the people of the east.}
\poetryline{They shall put out their hand against Edom and Moab,}
\poetryline{\hspace{2em}and the Ammonites shall obey them.}
\poetryline{\versenum{15} And the LORD will utterly destroy}
\poetryline{\hspace{2em}the tongue of the Sea of \highlightgray{Egypt},}
\poetryline{and will wave his hand over the River}
\poetryline{\hspace{2em}with his scorching \highlightorange{breath},}
\poetryline{and strike it into seven channels,}
\poetryline{\hspace{2em}and he will lead people across in sandals.}
\poetryline{\versenum{16} And there will be a highway from \highlightgray{Assyria}}
\poetryline{\hspace{2em}for \highlightblue{the remnant that remains of his people},}
\poetryline{as there was for Israel}
\poetryline{\hspace{2em}when they came up from the land of \highlightgray{Egypt}.}
\end{versesection}

\end{biblicaloutline}

\vspace{2em}

{\large\bfseries Tongue of Egypt into 7 Channels}
\vspace{1em}

The "tongue of the Sea of Egypt" is the Gulf of Suez — the narrow body of water extending from the Red Sea up along the Sinai Peninsula. God's scorching breath (\hebrew{רוּחַ}, \textit{ruach} — the same word for "Spirit" at the start of this chapter and the same word used in chapter 4 with the spirit/breath of fire) will break it into seven channels or wadis.

A \textit{wadi} is a dry riverbed in desert regions that only fills with water during seasonal rains. The imagery here suggests God will make the impassable sea as easy to cross as walking through dry desert channels.

\vspace{1em}

\begin{center}
\begin{minipage}[t]{0.45\textwidth}
\centering
\includegraphics[width=0.9\textwidth]{./gulf-of-suez-map.png}\\
\vspace{0.5em}
\textit{The "Tongue" of the Sea of Egypt\\(Gulf of Suez)}
\end{minipage}
\hspace{0.05\textwidth}
\begin{minipage}[t]{0.45\textwidth}
\centering
\includegraphics[width=0.9\textwidth]{./wadi-example.png}\\
\vspace{0.5em}
\textit{A Wadi\\Dry riverbed that fills during rains}
\end{minipage}
\end{center}

\vspace{2em}

{\large\bfseries The Second Exodus}
\vspace{1em}

There's going to be a new Exodus. Just as God brought Israel out of Egypt through the Red Sea, He will bring the remnant back from exile with even greater miraculous intervention. The reference to "as there was for Israel when they came up from the land of Egypt" (v.16) deliberately echoes the first exodus.

This sets up chapter 12, which mirrors the Song of the Sea from Exodus 15. After Moses led Israel through the Red Sea, they sang a song of thanksgiving and victory. After the Root of Jesse leads the remnant through the second exodus, they will sing a similar song:

\begin{quote}
\textit{"You will say in that day: 'I will give thanks to you, O LORD, for though you were angry with me, your anger turned away, that you might comfort me.'"}\\\\
\hfill --- Isaiah 12:1
\end{quote}

The end of chapter 11 flows seamlessly into chapter 12's song of praise, completing the pattern: judgment → deliverance → thanksgiving.

\vspace{2em}

\end{document}
