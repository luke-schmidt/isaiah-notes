\documentclass[11pt]{article}
\usepackage[margin=1in]{geometry}
\usepackage{../../styles/isaiah}
% Hebrew font configuration is now provided by the main style package

\begin{document}

% Isaiah 2:1-5 - ESV Text
\begin{center}
{\Huge\bfseries Isaiah 2:1-5}
\end{center}
\vspace{10pt}

\isaiahChiasticOverview{2}

\configurableGrid{1}{Chapters 2-4}{The Word of Isaiah (1)}{The Mountain of Yahweh (2-5)}{Streaming Nations}{Swords to Plowshares}

% Isaiah 2:1-5 - ESV Text
\begin{biblicaloutline}[Isaiah 2:1-5]
    
    \subsectionheader{The Word of Isaiah (1)}

    \begin{versesection}{2em}
        \versenum{1} The \highlightaqua{word} that Isaiah the son of Amoz saw concerning Judah and Jerusalem.
    \end{versesection}
    
    \subsectionheader{The Mountain of Yahweh (2-5)}
    
    \begin{versesection}{2em}
        \versenum{2} It shall come to pass in the latter days
        \poetryline that the \highlightgreen{mountain} of the house of the LORD
        shall be established as the highest of the \highlightgreen{mountains},
        \poetryline and shall be \highlightblue{lifted up} above the hills;
        and all the \highlightsilver{nations} shall flow to it,
        
        \versenum{3}\poetryline  and many peoples shall come, and say:
        ``Come, let us go up to the \highlightgreen{mountain} of the LORD,
        \poetryline to the house of the God of Jacob,
        that he may teach us his ways
        \poetryline and that we may walk in his paths.''
        For out of Zion shall go forth the law,
        \poetryline and the \highlightaqua{word} of the LORD from Jerusalem.
        \versenum{4} He shall judge between the \highlightsilver{nations},
        \poetryline and shall decide disputes for many peoples;
        and they shall beat their swords into plowshares,
        \poetryline and their spears into pruning hooks;
        \highlightsilver{nation} shall not \highlightblue{lift up} sword against \highlightsilver{nation},
        \poetryline neither shall they learn war anymore.
        
        \versenum{5} O house of Jacob,
        \poetryline come, let us walk
        \poetryline in the \highlightyellow{light} of the LORD.
    \end{versesection}

\end{biblicaloutline}

\newpage
{\large\bfseries Mountain of Yahweh}
\vspace{1em}

The mountain of Zion itself for sure isn't the tallest mountain in the Ancient Near East – it isn't even the tallest mountain in the range of mountains around it!

\vspace{1em}
So what is Isaiah getting at with the mountain being the highest above the hills?

\vspace{1em}

Mountains in the Ancient Near Eastern cultures signified a spot where heaven and earth met.
Think about famous stories where there are "mountaintop" moments throughout the Bible.

\vspace{1em}
One of the predominant examples is Moses on Mt. Sinai receiving the Law. It's almost as if this vision
of Mt. Zion where the nations stream up to receive the instruction of Yahweh is a second Sinai of sorts.
It's the ultimate "heaven and earth" spot.

\vspace{3em}
{\large\bfseries Streaming Nations}
\vspace{1em}

The word for "flow" in verse 2 is the verb form of the noun "river".
Here we have a reversal of Genesis 2 where instead of the river coming from an Eden-like mountain to water the nations,
the nations themselves are now flowing \textbf{to} the Eden-like mountain.
\vspace{1em}

\textit{``A river flowed out of Eden to water the garden, and there it divided and became four rivers. The name of the first is the Pishon. It is the one that flowed around the whole land of Havilah, where there is gold. And the gold of that land is good; bdellium and onyx stone are there. The name of the second river is the Gihon. It is the one that flowed around the whole land of Cush. And the name of the third river is the Tigris, which flows east of Assyria. And the fourth river is the Euphrates.''} — Genesis 2:10-14
\vspace{1em}

Here, in a seemingly abrupt passage in the Eden story, we see something vitally important for the purposes of understanding this section of Isaiah.
\vspace{1em}

The river of Eden splits off into 4 parts that each water a different area extremely relevant to the biblical story:
\begin{itemize}
    \item \textbf{Havilah} - Pishon (meaning: 'gusher') waters the area where Ishmael's descendants will settle (Gen. 25:18). Probably south and east of Canaan.
    \item \textbf{Cush} - Gihon (meaning: 'bursting forth') waters the land of Cush, where the Ethiopians and Egyptians would be.
    \item \textbf{Assyria} - Tigris flows and waters Assyria.
    \item \textbf{Babylon} - The Euphrates River doesn't even mention the name of where it flows, the readers would know.
\end{itemize}

So from this list we see all of the major players in the drama of Israel all experiencing the blessing flowing from the mountain of Eden.

In Isaiah 2, written to a people currently at odds or in hot water with most of the regions mentioned in Gen 2, we see a reversal where these nations are the ones
now "rivering" back up to the mountain of the LORD to receive his instruction.

It's the redemption of a lost Eden!


\vspace{3em}
{\large\bfseries Swords to Plowshares}
\vspace{1em}

Note how the weapons typically meant for war and destruction are turned into farming equipment
used for production and growth.

All of this flows from the nations rivering up to the mountain of Yahweh to learn his instruction.

{\vspace{1em}}

Contrary to the world's stereotypical view of the Old Testament, the output of learning God's Law and instructions
should be a people transformed from using instruments of destruction of life to instruments of cultivation of life.

\begin{thesauce}
\sauceitem{This passage is almost copy/paste of Micah 4:1-5. Who wrote it first? Did the LORD reveal it to both individually or did the prophets know each other?}

\sauceitem{How does this passage relate to the Tower of Babel? Who's making the Highest Mountain to the heavens?}
\end{thesauce}

\end{document}