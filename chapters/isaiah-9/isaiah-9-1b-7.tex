\documentclass[11pt]{article}
\usepackage[margin=1in]{geometry}
\usepackage{../../styles/isaiah}
\usepackage{fontspec}
\usepackage{bidi}
\usepackage{../../styles/components/leftRightGrid}
\begin{document}
% Isaiah Context Grid
\leftRightGrid{4}{Isaiah 6-9:7}{
    \textbf{\large Isaiah 6}

    \textit{Isaiah's Call \& \\Commission}
}{
    \textbf{\large Isaiah 7}

    \textit{Immanuel}
}{
    \textbf{\large Isaiah 8}

    \textit{Maher-shalal-hash-baz}
}
{
    \textbf{\large Isaiah 9:1-7}

    \textit{The Prince of Peace}
}

\newpage

% Main Section: Isaiah 9:1b-7
\begin{biblicaloutline}[Isaiah 9:1b-7]

    \subsectionheader{Joy Instead of Darkness (v1b-3)}
    \begin{versesection}{2em}
        \versenum{1b} In the former time he brought into contempt the land of Zebulun and the land of Naphtali, but in the latter time he has made glorious the way of the sea, the land beyond the Jordan, Galilee of the nations.
        {\vspace{1em}}
        {\versenum{2} The people who walked in \highlightsilver{darkness}}
        \poetryline{have seen a great \highlightyellow{light};}
        those who dwelt in a land of deep \highlightsilver{darkness},
        \poetryline{on them has \highlightyellow{light} shone.}

        {\versenum{3} You have multiplied the nation;}
        \poetryline{you have increased its joy;}
        {they rejoice before you}
        \poetryline{as with joy at the harvest,}
        \poetryline{as they are glad when they divide the spoil.}
    \end{versesection}

    \subsectionheader{Enemies Destroyed (v4-5)}
    \begin{versesection}{2em}
        {\versenum{4} For the yoke of his burden,}
        \poetryline{and the staff for his shoulder,}
        \poetryline{the rod of his oppressor,}
        \poetryline{you have broken as on the day of Midian.}

        {\versenum{5} For every boot of the tramping warrior} in \sectionwordfootnote{battle tumult}{Lit. "boot, booting with shaking"},
        \poetryline{and every garment rolled in blood}
        \poetryline{will be burned as fuel for the fire.}
    \end{versesection}

    \subsectionheader{Child is Born (v6-7)}
    \begin{versesection}{2em}
        {\versenum{6} For unto us a child is born,}
        \poetryline{to us a son is given;}
        {and the government shall be upon his shoulder,}
        \poetryline{and his name shall be called}
        {Wonderful Counselor, Mighty God,}
        \poetryline{Everlasting Father, Prince of Peace.}

        {\versenum{7} Of the increase of his government and of peace}
        \poetryline{there will be no end,}
        {on the throne of David and over his kingdom,}
        \poetryline{to establish it and to uphold it}
        {with \highlightaqua{justice} and with \highlightaqua{righteousness}}
        \poetryline{from this time forth and forevermore.}
        {The zeal of the LORD of hosts will do this.}
    \end{versesection}

\end{biblicaloutline}

\vspace{3em}
{\large\bfseries Matthew's Quotation of Isaiah 9:1b-2}
\vspace{1em}

Matthew 4:12-17 directly quotes this passage when describing Jesus's ministry in Galilee. [Placeholder: Add analysis of how Matthew applies this prophecy to Jesus's ministry, the significance of Galilee of the Gentiles, and the fulfillment of light shining in darkness.]

\vspace{1em}
\begin{center}
\includegraphics[width=.75\textwidth]{image.png}
\end{center}
\vspace{1em}

\vspace{3em}
{\large\bfseries The People Walking in Darkness}
\vspace{1em}

[Placeholder: Explain how the "people walking in darkness" refers to both the Northern Kingdom (v1b mentions Zebulun and Naphtali) and the Southern Kingdom (8:19-21 describes their darkness). Note how they are referred to as one unified nation in v3, anticipating reunification under the coming child King.]

\vspace{3em}
{\large\bfseries Comparison with Judges 6-8: Gideon and the Child}
\vspace{1em}

[Placeholder: Add comparison table showing parallels between Isaiah 6-9:7 and Judges 6-8]

\begin{comparisontable}{Isaiah 6-9:7}{Judges 6-8}{Element}

\verserow{
[Placeholder]
}{
[Placeholder]
}{
Sign Requested
}

\verserow{
[Placeholder]
}{
[Placeholder]
}{
"Insignificant" Leader
}

\verserow{
[Placeholder]
}{
[Placeholder]
}{
Victory over Oppressors
}

\verserow{
Day of Midian (9:4)
}{
Gideon defeats Midianites
}{
Deliverance Pattern
}

\end{comparisontable}

\vspace{3em}
{\large\bfseries "Prince of Peace" vs. "King"}
\vspace{1em}

[Placeholder: Discuss why the title uses "Prince" (\textit{sar}) instead of "King" (\textit{melek}). Scholar Carlson suggests "Prince" (\textit{sar}) may have been chosen because it is close to the Assyrian word for King (\textit{sarrum}), potentially communicating royal authority in a way that would resonate with the imperial context of Isaiah's audience.]

\end{document}