\documentclass[11pt]{article}
\usepackage[margin=1in]{geometry}
\usepackage{../../styles/isaiah}
\usepackage{../../styles/components/isaiahOraclesOverview}

\begin{document}

\newpage
\begin{center}
{\Huge\bfseries Isaiah 14:28-32}
\end{center}
\vspace{10pt}

% Oracle Overview for Chapters 13-23
\isaiahOraclesOverview{2}
\newpage
% Chapter 14:28-32
\begin{chiasticoutline}[Isaiah 14:28-32 — Oracle Against Philistia]{.95em}{2em}

\chiasticverse[A]{0}{\versenum{28} In the year that King Ahaz \highlightred{died} came this oracle:}

\chiasticverse[B]{1}{%
\versenum{29a} Rejoice not, \highlightyellow{O Philistia, all of you},

\poetryline that the rod that struck you is broken,
}

\chiasticverse[C]{2}{%
\versenum{29b} for from the serpent's \highlightgreen{root} will come forth an adder,

\poetryline and its fruit will be a flying \sectionwordfootnote{fiery serpent}{Seraph — same angels from Chapter 6}.
}

\chiasticverse[D]{3}{%
\versenum{30a} And the firstborn of the poor will \sectionwordfootnote{graze}{Line could also be translated, "the poor will graze in my pastures"},

\poetryline and the needy lie down in safety;
}

\chiasticverse[C']{2}{%
\versenum{30b} but I will \highlightred{kill} your \highlightgreen{root} with famine,

\poetryline and your remnant it will slay.
}

\chiasticverse[B']{1}{%
\versenum{31} Wail, O gate; cry out, O city;

\poetryline melt in fear, \highlightyellow{O Philistia, all of you}!

For smoke comes out of the north,

\poetryline and there is no straggler in his ranks.
}

\chiasticverse[A']{0}{%
\versenum{32} What will one answer the messengers of the nation?

\poetryline "The LORD has founded Zion,

\poetryline and in her the afflicted of his people find refuge."
}

\end{chiasticoutline}

{\vspace{4em}}
{\large\bfseries Who's the Rod?}
{\vspace{1em}}

This rather cryptic section has a lot of scholars divided on who each of the characters in this passage are. Most see significance in the dating of the prophecy in v28 in how it plays in with the rest of the passage.

{\vspace{1em}}

The rod that's broken though doesn't necessarily seem to be King Ahaz directly though, since Philistia was actually profitable under his rule (see 2 Chronicles 28:18). Likely what this is referring to is seemingly decaying Assyrian power around the same time as Ahaz's death. So it's not necessarily Ahaz as the rod, but Assyria's power around the same time as Ahaz's death.

\begin{quote}
\textit{"In those days the Philistines also had invaded the cities of the lowland and of the Negeb of Judah, and had taken Beth-shemesh, Aijalon, Gederoth, Soco with its villages, Timnah with its villages, and Gimzo with its villages. And they settled there."}\\
\hfill --- 2 Chronicles 28:18
\end{quote}

\newpage 

{\large\bfseries Who's the Serpent?}
{\vspace{1em}}

Given Assyria's apparent weaknesses and Philistia looking to capitalize, this oracle from Isaiah is letting them know that a larger snake-like power is still coming for them. Using the same "root" language earlier in Isaiah, he's now using it to refer to an Assyrian root leading to a seed of the snake (cf. Gen 3:16).

{\vspace{1em}}

One other view here is from Motyer who views the root in v29 and the fruit from it is all in reference to the Messiah who will lead the punishment of Philistia at the hand of Israel (cf. ch 11).

{\vspace{2em}}
{\large\bfseries Who's the Poor?}
{\vspace{1em}}

The poor of v30 could be a precursor to the afflicted in v32 in that Yahweh's people themselves have a home and security through Yahweh's power. Philistia seems to have been going to Judah at Ahaz's death and using it as an excuse to get Judah on board with an Assyrian revolt.

{\vspace{1em}}

The response of Isaiah is the same as his response to Ahaz during his life — don't trust in political alliances, trust in Yahweh Himself Who can save and give security to the most vulnerable and afflicted in society.

\end{document}
