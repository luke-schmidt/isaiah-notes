\documentclass[11pt]{article}
\usepackage[margin=1in]{geometry}
\usepackage{../../styles/isaiah}
\usepackage{../../styles/components/threeBoxGrid}

\begin{document}

% Overview of Isaiah 1-12 Chiastic Structure
\isaiahChiasticOverview{5}

% Isaiah Context Grid
\threeBoxGrid{2}{Isaiah 9:8-10:34}{
    \textbf{\large Isaiah 9:8-21}

    \textit{Anger not Turned \\Away from Israel}
}{
    \textbf{\large Isaiah 10:1-4}

    \textit{Woe to Israel \& Anger \\not Turned Away}
}{
    \textbf{\large Isaiah 10:5-34}

    \textit{Woes and Therefores \\to Assyria}
}


% Verses 1-4 - Woe to Israel
\begin{biblicaloutline}[Isaiah 10:1-4]

    \begin{versesection}{2em}
        \versenum{1} \highlightsilver{Woe} to those who decree iniquitous decrees,
        \poetryline and the writers who keep writing oppression,
        \versenum{2} to \sectionwordfootnote{turn aside the needy from justice}{Literally – "stretching out the needy from justice" the same word for God's hand being stretched out}
        \poetryline and to rob the poor of my people of their right,
        that \highlightpurple{widows} may be their spoil,
        \poetryline and that they may make the \highlightpurple{fatherless} their prey!
        \versenum{3} What will you do on the day of punishment,
        \poetryline in the ruin that will come from afar?
        To whom will you flee for help,
        \poetryline and where will you leave your \sectionwordfootnote{wealth}{Literally – "glory"}?
        \versenum{4} Nothing remains but to crouch among the prisoners
        \poetryline or fall among the slain.
        \highlightorange{For all this his anger has not turned away,}
        \poetryline \highlightorange{and his hand is stretched out still.}
    \end{versesection}

\end{biblicaloutline}
\newpage
\vspace{3em}
{\large\bfseries Robbing Poor of their Right}
\vspace{1em}

Isaiah's rebuke of Israel's leaders for robbing the poor and making widows and orphans their prey finds an echo in Jesus' confrontation with the religious leaders of His day:

\begin{quote}
\textit{"And in the hearing of all the people he said to his disciples, 'Beware of the scribes, who like to walk around in long robes, and love greetings in the marketplaces and the best seats in the synagogues and the places of honor at feasts, who devour widows' houses and for a pretense make long prayers. They will receive the greater condemnation.'"}\\
\hfill --- Luke 20:45-47
\end{quote}

\vspace{1em}

While Jesus doesn't quote Isaiah 10 directly here, it seems that this section of Scripture was on His mind for a few reasons.

Consider the language Isaiah has been using throughout chapter 9 and into chapter 10: nations \highlightblue{devouring} Israel with \highlightblue{open mouth} (9:12), leaders causing people to be \highlightblue{swallowed up} (9:16), wickedness \highlightblue{devouring} like fire (9:18), and tribes \highlightblue{devouring} one another (9:19-21). Now in 10:2, these same predatory leaders make widows and orphans their "spoil" and "prey" ("shalal" and "baz" from ch 8)

When Jesus warns that the scribes and Pharisees "\highlightblue{devour} widows' houses," He employs this same visceral, predatory language. The repetition of "devour" throughout Isaiah 9-10, culminating in the exploitation of the vulnerable, creates a thematic thread that Jesus appears to draw upon. Jesus confronted religious leaders who, like their ancient counterparts, used their positions to exploit rather than protect the most vulnerable—all while maintaining an outward show of piety with their "long prayers" and religious garments.

The pattern remains consistent across centuries: those entrusted with leading God's people can become their greatest predators, \highlightblue{devouring} the very ones they should defend.

\end{document}
