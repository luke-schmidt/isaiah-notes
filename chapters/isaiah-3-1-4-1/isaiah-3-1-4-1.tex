\documentclass[11pt]{article}
\usepackage[margin=1in]{geometry}
\usepackage{../../styles/isaiah}

\begin{document}

\begin{center}
{\Huge\bfseries Isaiah 3:1-4:1 – Leadership Crisis}
\end{center}

% Overview of Isaiah 1:2-5:30 with section 3:1-4:1 bolded
\isaiahOverviewCombined{4}

{\vspace{3em}}

% Overview of Isaiah 3:1-4:1
\begin{overview}{Isaiah 3:1-4:1 — Overview}

\overviewsection[0]{%
\textbf{Verses 1-7: God Taking Away (Men)}
\begin{itemize}
    \item All supports and leaders removed from Jerusalem and Judah
    \item Boys and infants will rule, causing social chaos
    \item Leadership crisis and refusal to lead
\end{itemize}
}

\overviewsection[1]{%
\textbf{Verses 8-15: Why God is Taking Away (Men)}
\begin{itemize}
    \item Jerusalem stumbled because they defy God's glorious presence
    \item Leaders have oppressed the poor and devoured the vineyard
    \item Divine judgment pronounced against injustice
\end{itemize}
}

\overviewsection[1]{%
\textbf{Verses 16-17: Why God is Taking Away (Women)}
\begin{itemize}
    \item Daughters of Zion are haughty and vain
    \item Their pride will be brought low through humiliation
\end{itemize}
}

\overviewsection[0]{%
\textbf{Verses 18-4:1: God Taking Away (Women)}
\begin{itemize}
    \item Removal of all finery and luxury items
    \item Beauty replaced with shame and loss
    \item Desperation leads to reversing social norms
\end{itemize}
}

\end{overview}

% A - Verses 1-7: God taking away from the men
\begin{biblicaloutline}[Isaiah 3:1-7]

\subsectionheader{All Supports Removed (1-3)}

\begin{versesection}{2em}
\versenum{1} For behold, the \highlightgreen{Lord GOD of hosts}
\poetryline \highlightgreen{is taking away} from Jerusalem and from Judah
support and supply,
\poetryline all support of \highlightorange{bread},
\poetryline and all support of water;

\versenum{2} the mighty man and the soldier,
\poetryline the judge and the prophet,
\poetryline the diviner and the elder,

\versenum{3} the captain of fifty
\poetryline and the man of rank,
the counselor and the skillful magician
\poetryline and the expert in charms.
\end{versesection}

\subsectionheader{Children Will Rule (4-5)}

\begin{versesection}{2em}
\versenum{4} And I will make boys their princes,
\poetryline and infants shall rule over them.

\versenum{5} And the people will oppress one another,
\poetryline every one his fellow
\poetryline and every one his neighbor;
the youth will be insolent to the elder,
\poetryline and the despised to the honorable.
\end{versesection}

\subsectionheader{Leadership Crisis (6-7)}

\begin{versesection}{2em}
\versenum{6} For \highlightblue{a man will take hold of his brother}
\poetryline in the house of his father, saying:
"You have a cloak;
\poetryline you shall be our leader,
\poetryline and this heap of ruins
\poetryline shall be under your rule."

\versenum{7} In that day he will speak out, saying:
\poetryline "I will not be a healer;
in my house there is neither \highlightorange{bread} nor cloak;
\poetryline you shall not make me
\poetryline leader of the people."
\end{versesection}

\end{biblicaloutline}

\vspace{3em}
{\large\bfseries Leader -> Healer}
\vspace{1em}

In verses 6-7 we see a clear comparison between being the leader of Israel and being a "healer" – \hebrew{חָבַשׁ} (cha.vash) – literally
meaning someone who binds up. It's the same usage in 1:6 about there being sores abounding in Israel with no one to do this healing work.

{\vspace{1em}}

It's not until 61:1 that the full picture of this healing leader that's needed comes in to view in the passage Jesus read from
when kicking off his ministry.

{\vspace{1em}}

\begin{quote}
\textit{"The Spirit of the Lord GOD is upon me,\\ because the LORD has anointed me\\to bring good news to the poor;\\ he has sent me to bind up the brokenhearted,\\ to proclaim liberty to the captives,\\ and the opening of the prison to those who are bound;"}\\\\
\hfill --- Isaiah 61:1
\end{quote}

% B - Verses 8-15: Why God is taking away
\begin{biblicaloutline}[Isaiah 3:8-15]

\subsectionheader{Jerusalem's Defiance (8-9a)}

\begin{versesection}{2em}
\versenum{8} For Jerusalem has stumbled,
\poetryline and Judah has fallen,
because their speech and their deeds are against the LORD,
\poetryline defying his glorious presence.

\versenum{9a} For the look on their faces bears witness against them;
\poetryline they proclaim their sin like Sodom;
\poetryline they do not hide it.
\end{versesection}

\subsectionheader{Righteous vs Wicked (9b-12)}

\begin{versesection}{2em}
\versenum{9b} Woe to them!
\poetryline For they have brought \highlightred{evil} on themselves.

\versenum{10} Tell the righteous that it shall be \highlightyellow{well} with them,
\poetryline for they shall \highlightpurple{eat the fruit} of their deeds.

\versenum{11} Woe to the wicked! It shall be \highlightred{ill} with them,
\poetryline for what their hands have dealt out shall be done to them.

\versenum{12} \highlightsilver{My people}—infants are their oppressors,
\poetryline and women rule over them.
O \highlightsilver{my people}, your guides mislead you
\poetryline and they have swallowed up the course of your paths.
\end{versesection}

\subsectionheader{Divine Judgment Against Leaders (13-15)}

\begin{versesection}{2em}
\versenum{13} The LORD has taken his place to contend;
\poetryline he stands to judge peoples.

\versenum{14} The LORD will enter into judgment
\poetryline with the elders of his people and its princes:
"It is you who have devoured the vineyard,
\poetryline the spoil of the poor is in your houses.

\versenum{15} What do you mean by crushing my people,
\poetryline by grinding the face of the poor?"
declares the Lord GOD of hosts.
\end{versesection}

\end{biblicaloutline}

{\large\bfseries Echoes of Chapter 1}
\vspace{1em}

This passage strongly echoes themes from Isaiah 1, often using specific words that only show up in either chapter 1 or this chapter. This shows the continuing pattern of God's judgment on Jerusalem and Judah.

\begin{comparisontable}{Isaiah 1}{Isaiah 3}{Notes}

\verserow{
\versenum{6} From the sole of the foot even to the head,\\
\hspace*{2em} there is no soundness in it,\\
but bruises and sores\\
\hspace*{2em} and raw wounds;\\
they are not pressed out or \highlightaqua{bound up}\\
\hspace*{2em} or softened with oil.
}{
\versenum{7} In that day he will speak out, saying:\\
\hspace*{2em} "I will not be a \highlightaqua{healer}";
}{
Need for Healing
}

\verserow{
\versenum{9} we should have been like \highlightred{Sodom},\\
\hspace*{2em} and become like Gomorrah.\\
\versenum{10} Hear the word of the LORD, you rulers of \highlightred{Sodom}!
}{
\versenum{9} they proclaim their sin like \highlightred{Sodom};\\
\hspace*{2em} they do not hide it.
}{
Sodom Comparison
}

\verserow{
\versenum{19} If you are willing and obedient,\\
\hspace*{2em} you shall \highlightpurple{eat the} \highlightyellow{good} of the land;\\
\versenum{20} but if you refuse and rebel,\\
\hspace*{2em} you shall be eaten by the sword
}{

\versenum{9b} Woe to them!\\
\hspace*{2em} For they have brought \highlightred{evil} on themselves.\\
\versenum{10} Tell the righteous that it shall be \highlightyellow{well} with them,\\
\hspace*{2em} for they shall \highlightpurple{eat the fruit} of their deeds.\\
\versenum{11} Woe to the wicked! It shall be \highlightred{ill} with them
}{
Good vs Evil Outcomes
}

\verserow{
\versenum{17} learn to do good; seek justice,\\
\hspace*{2em} correct oppression;\\
bring justice \boldblue{to the fatherless},\\
\hspace*{2em} \boldblue{plead the widow's cause}.
}{
\versenum{14-15} "It is you who have devoured the vineyard,\\
\hspace*{2em} \boldblue{the spoil of the poor} is in your houses.\\
What do you mean by \boldblue{crushing my people},\\
\hspace*{2em} by \boldblue{grinding the face of the poor}?"
}{
Injustice Against the Poor
}

\verserow{
\versenum{3} The ox knows its owner,\\
\hspace*{2em} and the donkey its master's crib,\\
but Israel does not know,\\
\hspace*{2em} \highlightsilver{my people} do not understand.
}{
\versenum{12} \highlightsilver{My people}—infants are their oppressors,\\
\hspace*{2em} and women rule over them.\\
O \highlightsilver{my people}, your guides mislead you\\
\hspace*{2em} and they have swallowed up the course of your paths.
}{
"My People" - Failed Understanding
}

\verserow{
\versenum{20} but if you refuse and rebel,\\
\hspace*{2em} you shall be eaten by \highlightyellow{the sword};\\
\hspace*{2em} for the mouth of the LORD has spoken.
}{
\versenum{25} Your men shall fall by \highlightyellow{the sword}\\
\hspace*{2em} and your mighty men in battle.
}{
Death by the Sword
}

\end{comparisontable}


\vspace{5em}
{\large\bfseries Sin of Sodom}
\vspace{1em}

What is the "sin like Sodom" in this passage that's being fleshed out? How does Isaiah see what Israel's sin is like
and why would he compare it to Sodom?

{\vspace{1em}}
Most of the prophets when referring to the sins of Sodom and Gomorrah see something
more than just sexual immorality – it was a neglect and care for the outsider.

\begin{quote}
\textit{"Behold, this was the iniquity of thy sister Sodom,\\
pride, fulness of bread, and abundance of idleness was in her and in her daughters,\\
neither did she strengthen the hand of the poor and needy.\\
And they were haughty, and committed abomination before me:\\
therefore I took them away as I saw good."}\\\\
\hfill --- Ezekiel 16:49-50
\end{quote}

{\vspace{1em}}

% B' - Verses 16-17: Why God is taking away (women)
\begin{biblicaloutline}[Isaiah 3:16-17]

\subsectionheader{Haughty Daughters of Zion}

\begin{versesection}{2em}
\versenum{16} The LORD said:
Because the daughters of Zion are haughty
\poetryline and walk with outstretched necks,
\poetryline glancing wantonly with their eyes,
mincing along as they go,
\poetryline tinkling with their feet,

\versenum{17} therefore the Lord will strike with a scab
\poetryline the heads of the daughters of Zion,
\poetryline and the LORD will lay bare their secret parts.
\end{versesection}

\end{biblicaloutline}

% A' - Verses 18-4:1: God taking away from the women
\begin{biblicaloutline}[Isaiah 3:18-4:1]

\subsectionheader{Removal of 21 Items (18-23)}

\begin{versesection}{2em}
\versenum{18} In that day \highlightgreen{the Lord will take away}
the finery of the anklets, the headbands, and the crescents;

\versenum{19} the pendants, the bracelets, and the scarves;

\versenum{20} the headdresses, the armlets, the sashes,
\poetryline the perfume boxes, and the amulets;

\versenum{21} the signet rings and nose rings;

\versenum{22} the festal robes, the mantles, the cloaks,
\poetryline and the handbags;

\versenum{23} the mirrors, the linen garments,
\poetryline the turbans, and the veils.
\end{versesection}

\subsectionheader{Beauty Replaced with Shame (24-26)}

\begin{versesection}{2em}
\versenum{24} Instead of perfume there will be rottenness;
\poetryline and instead of a belt, a rope;
and instead of well-set hair, baldness;
\poetryline and instead of a rich robe, a skirt of sackcloth;
\poetryline and branding instead of beauty.

\versenum{25} Your men shall fall by \highlightyellow{the sword}
\poetryline and your mighty men in battle.

\versenum{26} And her gates shall lament and mourn;
\poetryline empty, she shall sit on the ground.
\end{versesection}

\subsectionheader{Desperation and Role Reversal (4:1)}

\begin{versesection}{2em}
\versenum{1} And \highlightblue{seven women shall take hold of one man} in that day, saying,
"We will eat our own \highlightorange{bread} and wear our own clothes,
\poetryline only let us be called by your name;
\poetryline \highlightgreen{take away} our reproach."
\end{versesection}

\end{biblicaloutline}

\vspace{1em}
{\large\bfseries Taking Away Reproach}
\vspace{1em}

Again we have hope in men which the end of chapter 2 specifically told us not to do!

Where else can Israel find hope for this? This, again, leads the reader to keep going when they reared
in chapter 25:8 that there is some ultimate hope when it's Yahweh Himself who will be the one to take away reproach, not men.

{\vspace{2em}}

\begin{quote}
\textit{"He will swallow up death forever;\\ and the Lord GOD will wipe away tears from all faces,\\ and the reproach of his people he will take away from all the earth,\\ for the LORD has spoken."}\\\\
\hfill --- Isaiah 25:8
\end{quote}

\end{document}