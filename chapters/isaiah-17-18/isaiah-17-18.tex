\documentclass[11pt]{article}
\usepackage[margin=1in]{geometry}
\usepackage{../../styles/isaiah}
\usepackage{../../styles/components/isaiahOraclesOverview}

\begin{document}

\newpage
\begin{center}
{\Huge\bfseries Isaiah 17-18}
\end{center}
\vspace{10pt}

% Oracle Overview for Chapters 13-23
\isaiahOraclesOverview{4}

% Overview of Isaiah 17-18
\begin{overview}{Isaiah 17-18 — Overview}

\overviewsection[0]{%
\textbf{17:1-6: Intro/``In that Day'' \#1}
\begin{itemize}
    \item Judgment on Damascus and Ephraim
\end{itemize}
}

\overviewsection[1]{%
\textbf{17:7-8: ``In that Day'' \#2}
\begin{itemize}
    \item Turning to God rather than idols
\end{itemize}
}

\overviewsection[0]{%
\textbf{17:9-11: ``In that Day'' \#3}
\begin{itemize}
    \item Forgotten God and false strongholds
\end{itemize}
}
\vspace{40pt}
\overviewsection[0]{%
\textbf{17:12-14: ``Ah!'' \#1 and Judgment ``at that time''}
\begin{itemize}
    \item Rebuking Assyria and the nations
\end{itemize}
}

\overviewsection[0]{%
\textbf{18:1-7: ``Ah!'' \#2 and Judgment ``at that time''}
\begin{itemize}
    \item Judgment and hope for Cush
\end{itemize}
}

\end{overview}

% Chapter 17:1-6
\begin{biblicaloutline}[Isaiah 17:1-6 — Intro/``In that Day'' \#1]

\begin{versesection}{2em}
\versenum{1} An oracle concerning \highlightyellow{Damascus}.

Behold, \highlightyellow{Damascus} will cease to be a city
\poetryline and will become a \sectionwordfootnote{heap}{Wordplay with city. heap - `me-i'; city - `me-ir'} of ruins.
\versenum{2} The \highlightred{cities} of \sectionwordfootnote{Aroer}{Septuagint: "Cities that are forever deserted"} are \highlightblue{deserted};
\poetryline they will be for flocks,
\poetryline which will lie down, and none will make them afraid.
\versenum{3} The fortress will disappear from Ephraim,
\poetryline and the \sectionwordfootnote{kingdom}{lit. `kingship'/`sovereignty'} from \highlightyellow{Damascus};
and the \highlightgreen{remnant} of Syria will be
\poetryline like the glory of the children of Israel,
\poetryline \highlightpurple{declares the LORD of hosts}.

\versenum{4} \highlightpink{And in that day} the \sectionwordfootnote{glory of Jacob}{c.f. 1 Sam 4:21 - glory is taken away `Ichabod'/ `ich - ka'vod'} will be brought low,
\poetryline and the fat of his flesh will grow lean.
\versenum{5} And it shall be as when the \highlightorange{reaper} gathers standing grain
\poetryline and his arm \highlightorange{harvests} the ears,
and as when one gleans the ears of grain
\poetryline in the Valley of \sectionwordfootnote{Rephaim}{means `Ghost Valley' - c.f. ch 14:9}.
\versenum{6} Gleanings will be \highlightgreen{left} in it,
\poetryline as when an olive tree is beaten—
two or three berries
\poetryline in the top of the highest bough,
four or five
\poetryline on the branches of a fruit tree,
\poetryline \highlightpurple{declares the LORD God of Israel}.
\end{versesection}

\end{biblicaloutline}

{\vspace{4em}}
{\large\bfseries Why Damascus?}
{\vspace{1em}}

As we read in ch 7, Damascus and the Northern Kingdom of Israel (Ephraim) are united against Judah since they didn't want to band together with them against Assyria. This oracle, as we see in the verses right after, is more general and widespread than just Damascus. This pattern is consistent with the other oracles we've read so far too.

{\vspace{4em}}
{\large\bfseries For the Poor}
{\vspace{1em}}

The law of gleaning was meant to provide for the poor, widows, and sojourners:

\begin{quote}
\textit{``When you reap your harvest in your field and forget a sheaf in the field, you shall not go back to get it. It shall be for the sojourner, the fatherless, and the widow, that the LORD your God may bless you in all the work of your hands. When you beat your olive trees, you shall not go over them again. It shall be for the sojourner, the fatherless, and the widow. When you gather the grapes of your vineyard, you shall not strip it afterward. It shall be for the sojourner, the fatherless, and the widow. You shall remember that you were a slave in the land of Egypt; therefore I command you to do this.'' (Deuteronomy 24:19-22)}
\end{quote}

This law was surely top of mind for Isaiah when writing this. It's as if they will be like the poor they neglected in the first place!

% Chapter 17:7-8
\begin{biblicaloutline}[Isaiah 17:7-8 — ``In that Day'' \#2]

\begin{versesection}{2em}
\versenum{7} \highlightpink{In that day} man will \highlightteal{look} to his \highlightlightgray{Maker}, and his eyes will \highlightteal{look} on the Holy One of Israel.
\versenum{8} He will not \highlightteal{look} to the altars, the \highlightlightgray{work} of his hands, and he will not \highlightteal{look} on what his own fingers have \highlightlightgray{made}, either the Asherim or the altars of \sectionwordfootnote{incense}{only references are in reference to pagan worship}.
\end{versesection}

\end{biblicaloutline}

{\vspace{4em}}
{\large\bfseries Work of Their Hands}
{\vspace{1em}}

As we read in chapter 2:6-22, Israel very much put stock in what they possessed:

\begin{quote}
\textit{``Their land is filled with silver and gold, and there is no end to their treasures; their land is filled with horses, and there is no end to their chariots. Their land is filled with idols; they bow down to the work of their hands, to what their own fingers have made.'' (Isaiah 2:7-8)}
\end{quote}

The pattern of what happens to folks who build their own empire and rely upon it is consistent throughout all of Scripture. God will be worshipped one way or another.

How is this pattern taking place in our world today? Military might? Riches? AI?

% Chapter 17:9-11
\begin{biblicaloutline}[Isaiah 17:9-11 — ``In that Day'' \#3]

\begin{versesection}{2em}
\versenum{9} \highlightpink{In that day} their \highlightbrown{strong} \highlightred{cities} will be like the \highlightblue{deserted places} of the wooded heights and the \sectionwordfootnote{hilltops}{wooded heights and hilltops could be the names of cities (Horesh and Amir)}, which they \highlightblue{deserted} because of the children of Israel, and there will be desolation.

\versenum{10} For you have forgotten the God of your \sectionwordfootnote{salvation}{Isaiah's name}
\poetryline and have not remembered the Rock of your \highlightbrown{refuge};
therefore, though you plant pleasant plants
\poetryline and sow the vine-branch of a stranger,
\versenum{11} though you make them grow on the day that you plant them,
\poetryline and make them \highlightlime{blossom} in the morning that you sow,
yet the \highlightorange{harvest} will flee away
\poetryline in a day of grief and incurable pain.
\end{versesection}

\end{biblicaloutline}

{\vspace{4em}}
{\large\bfseries Real Stronghold}
{\vspace{1em}}

Compare Isaiah's usage of the Hebrew word for strong/refuge:

\hebrewword{Refuge/Strong}{מָעוֹז}{ma'oz}{A place of safety and protection; a stronghold or fortress}

Are they placing hope in their cities or their Rock of Salvation? 

Cities have an interesting history throughout the Bible too. Starting with the first city—Cain's—the picture of them is consistently one of folks building their own sense of security rather than trusting God's:

\begin{quote}
\textit{``Cain knew his wife, and she conceived and bore Enoch. When he built a city, he called the name of the city after the name of his son, Enoch.'' (Genesis 4:17)}
\end{quote}

Even cities are ultimately redeemed in the end though in the New Heavens and the New Earth:

\begin{quote}
\textit{``Then I saw a new heaven and a new earth, for the first heaven and the first earth had passed away, and the sea was no more. And I saw the holy city, new Jerusalem, coming down out of heaven from God, prepared as a bride adorned for her husband. And I heard a loud voice from the throne saying, `Behold, the dwelling place of God is with man. He will dwell with them, and they will be his people, and God himself will be with them as their God. He will wipe away every tear from their eyes, and death shall be no more, neither shall there be mourning, nor crying, nor pain anymore, for the former things have passed away.'... And I saw no temple in the city, for its temple is the Lord God the Almighty and the Lamb. And the city has no need of sun or moon to shine on it, for the glory of God gives it light, and its lamp is the Lamb.'' (Revelation 21:1-4, 22-23)}
\end{quote}

{\vspace{4em}}
{\large\bfseries Forgotten God}
{\vspace{1em}}

The danger of forgetting God is clearly warned about in Deuteronomy:

\begin{quote}
\textit{``Take care lest you forget the LORD your God by not keeping his commandments and his rules and his statutes, which I command you today, lest, when you have eaten and are full and have built good houses and live in them, and when your herds and flocks multiply and your silver and gold is multiplied and all that you have is multiplied, then your heart be lifted up, and you forget the LORD your God, who brought you out of the land of Egypt, out of the house of slavery, who led you through the great and terrifying wilderness, with its fiery serpents and scorpions and thirsty ground where there was no water, who brought you water out of the flinty rock, who fed you in the wilderness with manna that your fathers did not know, that he might humble you and test you, to do you good in the end. Beware lest you say in your heart, `My power and the might of my hand have gotten me this wealth.' You shall remember the LORD your God, for it is he who gives you power to get wealth, that he may confirm his covenant that he swore to your fathers, as it is this day. And if you forget the LORD your God and go after other gods and serve them and worship them, I solemnly warn you today that you shall surely perish. Like the nations that the LORD makes to perish before you, so shall you perish, because you would not obey the voice of the LORD your God.'' (Deuteronomy 8:11-20)}
\end{quote}

Forgetting God is apostasy that leads to idol worship.

{\vspace{4em}}
{\large\bfseries Planting Plants?}
{\vspace{1em}}

The ``Pleasant Plants'' could be referring to Tammuz—a pagan god worshipped by the nations at this time. The Greek translation of the Hebrew Bible (Septuagint) also translates this as ``Unfaithful plant and Unfaithful seed.''

Recall Isaiah 1:29:

\begin{quote}
\textit{``For they shall be ashamed of the oaks that you desired; and you shall blush for the gardens that you have chosen.''}
\end{quote}

% Chapter 17:12-14
\begin{biblicaloutline}[Isaiah 17:12-14 — ``Ah!'' \#1 and Judgment ``at that time'']

\begin{versesection}{2em}
\versenum{12} \highlightpink{Ah}, the thunder of many \highlightcyan{peoples};
\poetryline they thunder like the thundering of the \sectionwordfootnote{\highlightcyan{sea}}{Wordplay with people. sea - `yam'; people - `am'}!
Ah, the roar of nations;
\poetryline they roar like the roaring of mighty waters!
\versenum{13} The nations roar like the roaring of many waters,
\poetryline but he will rebuke them, and they will flee far away,
chased like chaff on the \highlightsilver{mountains} before the wind
\poetryline and whirling \sectionwordfootnote{dust}{could also be a tumbleweed or a type of wild artichoke that rolls} before the storm.
\versenum{14} \highlightcoral{At evening time}, behold, terror!
\poetryline Before morning, they are no more!
This is the portion of those who loot us,
\poetryline and the lot of those who plunder us.
\end{versesection}

\end{biblicaloutline}

{\vspace{4em}}
{\large\bfseries Like Chaff; Like Waters}
{\vspace{1em}}

The imagery of waters recalls chapter 8:7:

\begin{quote}
\textit{``therefore, behold, the Lord is bringing up against them the waters of the River, mighty and many, the king of Assyria and all his glory. And it will rise over all its channels and go over all its banks.''}
\end{quote}

When the husks have been separated from the grains of wheat or barley, the mixture is tossed into the air. The chaff is blown away while the heavier grain falls back to the ground. For this reason threshing floors were located in the most windy places, the hilltops (cf. 29:5).

\newpage
% Chapter 18:1-7
\begin{chiasticoutline}[Isaiah 18:1-7 — ``Ah!'' \#2 and Judgment ``at that time'']{.999999999em}{2em}

\chiasticverselabel[A]{0}{
\versenum{1} \highlightpink{Ah}, land of whirring wings\\
\poetryline that is beyond the rivers of Cush,\\
\versenum{2} which sends ambassadors by the \highlightcyan{sea},\\
\poetryline in vessels of papyrus on the waters!\\
Go, you swift messengers,\\
\poetryline \underline{to a nation tall and smooth,}\\
\poetryline \underline{to a \highlightcyan{people} feared near and far,}\\
\underline{a nation mighty and conquering,}\\
\poetryline \underline{whose land the rivers divide.}\\
}{}

\chiasticverse[B]{1}{
\versenum{3} All you inhabitants of the world,\\
\poetryline you who dwell on the \highlightolive{earth},\\
when a signal is raised on the \highlightsilver{mountains}, look!\\
\poetryline When a trumpet is blown, hear!\\
}

\chiasticverse[C]{2}{
\versenum{4} For thus the LORD said to me:\\
\poetryline "I will quietly look from my dwelling\\
like clear heat in sunshine,\\
\poetryline like a cloud of dew in the heat of \highlightorange{harvest}."\\
}

\chiasticverse[C']{2}{
\versenum{5} For before the \highlightorange{harvest}, when the \highlightlime{blossom} is over,\\
\poetryline and the flower becomes a ripening grape,\\
he cuts off the shoots with pruning hooks,\\
\poetryline and the spreading branches he lops off and clears away.\\
}

\chiasticverse[B']{1}{
\versenum{6} They shall all of them be \highlightblue{left}\\
\poetryline to the \sectionwordfootnote{birds of prey}{c.f. Ez 39:17-29 and Rev 19:17-18} of the \highlightsilver{mountains}\\
\poetryline and to the beasts of the \highlightolive{earth}.\\
And the birds of prey will summer on them,\\
\poetryline and all the beasts of the \highlightolive{earth} will \sectionwordfootnote{winter}{`Summer' and `Winter' indicates the bodies are rotting for a long time. No burial was a disgrace in the Ancient Near East. cf. Deut 28:26} on them.\\
}

\chiasticverselabel[A']{0}{
\versenum{7} \highlightcoral{At that time} tribute will be brought to the LORD of hosts\\
\underline{from a \sectionwordfootnote{\highlightcyan{people}}{`Nation' in v2 changes to `People' on the second iteration} tall and smooth,}\\
\poetryline \underline{from a \highlightcyan{people} feared near and far,}\\
\underline{a nation mighty and conquering,}\\
\poetryline \underline{whose land the rivers divide,}\\
to \highlightsilver{Mount Zion}, the place of the name of the LORD of hosts.\\
}{}

\end{chiasticoutline}

{\vspace{4em}}
{\large\bfseries Whirring Wings?}
{\vspace{1em}}

This could be because there were a lot of bugs in this area south of Egypt, but it could also be describing messengers that move so fast in the sea like insects. The Septuagint seems to say the same thing by translating this ``land of winged boats.''

{\vspace{4em}}
{\large\bfseries Who are the Ambassadors?}
{\vspace{1em}}

These could have been from a new dynasty in 715 BC that was Cushite. The ruling family could have sent envoys to Judah, Philistia, and Moab to revolt against Assyria...

{\vspace{4em}}
{\large\bfseries Who are they giving a message to?}
{\vspace{1em}}

...Or it could be Isaiah telling the messengers to go straight to Assyria themselves? Egypt isn't usually described this way... could be Medes? Folks aren't sure. ``Mighty''—literally a ``qaw-qaw'' nation—Hebrew gibberish to describe the people of Cush? The NIV translation may capture this better as a nation of ``strange speech.''

\newpage
\begin{biblecomparison}{Isaiah 18:2}
\translation{ESV}{which sends ambassadors by the sea, in vessels of papyrus on the waters! Go, you swift messengers, to a nation tall and smooth, to a people feared near and far, \textbf{a nation mighty and conquering}, whose land the rivers divide.}
\translation{NASB}{Which sends messengers by the sea, Even in papyrus vessels on the surface of the waters. Go, swift messengers, to a nation tall and smooth, To a people feared far and wide, \textbf{A powerful and oppressive nation} Whose land the rivers divide.}
\translation{NIV}{which sends envoys by sea in papyrus boats over the water. Go, swift messengers, to a people tall and smooth-skinned, to a people feared far and wide, \textbf{an aggressive nation of strange speech}, whose land is divided by rivers.}
\translation{KJV}{That sendeth ambassadors by the sea, even in vessels of bulrushes upon the waters, saying, Go, ye swift messengers, to a nation scattered and peeled, to a people terrible from their beginning hitherto; \textbf{a nation meted out and trodden down}, whose land the rivers have spoiled!}
\translation{NET}{that sends messengers by sea, in papyrus boats over the water! Go, you swift messengers, to a nation of tall, smooth-skinned people, to a people that are feared far and wide, \textbf{to a nation strong and victorious}, whose land rivers divide.}
\end{biblecomparison}

\vspace{1em}

The Hebrew word translated as ``mighty'' is particularly interesting:

\hebrewword{Mighty}{קַו-קָו}{qav-qav}{A nation of repetitive, unintelligible speech; literally ``line-line'' or ``measuring line-measuring line.'' Often interpreted as gibberish or a mocking reference to incomprehensible language. The repetition suggests either foreign speech that sounds like babbling to Hebrew ears, or possibly refers to their methodical, measured conquest (like measuring with a line).}

\newpage
{\large\bfseries Yahweh is Quietly Looking}
{\vspace{1em}}

Consider the parable of the wheat and tares in Matthew 13:24-30:

\begin{quote}
\textit{``He put another parable before them, saying, `The kingdom of heaven may be compared to a man who sowed good seed in his field, but while his men were sleeping, his enemy came and sowed weeds among the wheat and went away. So when the plants came up and bore grain, then the weeds appeared also. And the servants of the master of the house came and said to him, "Master, did you not sow good seed in your field? How then does it have weeds?" He said to them, "An enemy has done this." So the servants said to him, "Then do you want us to go and gather them?" But he said, "No, lest in gathering the weeds you root up the wheat along with them. Let both grow together until the harvest, and at harvest time I will tell the reapers, `Gather the weeds first and bind them in bundles to be burned, but gather the wheat into my barn.'"'}
\end{quote}

Folks are expecting big acts of judgment. The Lord is quietly looking.

{\vspace{4em}}
{\large\bfseries River of Cush}
{\vspace{1em}}

The rivers of Cush recall the rivers from the Garden of Eden in Genesis 2:10-14:

\begin{quote}
\textit{``A river flowed out of Eden to water the garden, and there it divided and became four rivers. The name of the first is the Pishon. It is the one that flowed around the whole land of Havilah, where there is gold. And the gold of that land is good; bdellium and onyx stone are there. The name of the second river is the Gihon. It is the one that flowed around the whole land of Cush. And the name of the third river is the Tigris, which flows east of Assyria. And the fourth river is the Euphrates.''}
\end{quote}

This is one of the nations that in Chapter 2 of Isaiah will ``river'' back up. Here in verse 7 we see that direct fulfillment!

\newpage
{\large\bfseries Verse 7 Fulfillment}
{\vspace{1em}}

Where else have we heard of people coming to know the Lord from Cush? Two specific instances come to mind:

\vspace{1em}
\textbf{Old Testament—Queen of Sheba:}

\begin{quote}
\textit{``Now when the queen of Sheba heard of the fame of Solomon concerning the name of the LORD, she came to test him with hard questions. She came to Jerusalem with a very great retinue, with camels bearing spices and very much gold and precious stones. And when she came to Solomon, she told him all that was on her mind... And she gave the king 120 talents of gold, and a very great quantity of spices and precious stones. Never again came such an abundance of spices as these that the queen of Sheba gave to King Solomon.'' (1 Kings 10:1-2, 10)}
\end{quote}

Coming to bring tribute to the Lord's anointed!

\vspace{1em}
\textbf{New Testament—Ethiopian Eunuch:}

\begin{quote}
\textit{``Now an angel of the Lord said to Philip, `Rise and go toward the south to the road that goes down from Jerusalem to Gaza.' This is a desert place. And he rose and went. And there was an Ethiopian, a eunuch, a court official of Candace, queen of the Ethiopians, who was in charge of all her treasure. He had come to Jerusalem to worship and was returning, seated in his chariot, and he was reading the prophet Isaiah.'' (Acts 8:26-28)}
\end{quote}

This is all fulfillment of Psalm 68:29 and Psalm 76:11:

\begin{quote}
\textit{``Because of your temple at Jerusalem kings shall bear gifts to you.'' (Psalm 68:29)}

\textit{``Let all around him bring gifts to him who is to be feared.'' (Psalm 76:11)}
\end{quote}

\end{document}
