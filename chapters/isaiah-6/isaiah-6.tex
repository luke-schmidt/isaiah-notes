\documentclass[11pt]{article}
\usepackage[margin=1in]{geometry}
\usepackage{../../styles/isaiah}

\begin{document}

% Isaiah 6 - ESV Text
\begin{center}
{\Huge\bfseries Isaiah 6}
\end{center}
\vspace{10pt}

% Overview of Isaiah 6
\begin{overview}{Isaiah 6 — Overview}

\overviewsection[0]{%
\textbf{Verses 1-5: Vision of the Holy King}
\begin{itemize}
    \item Isaiah sees the LORD on His throne with seraphim crying "Holy, holy, holy"
    \item Isaiah recognizes his uncleanness in the presence of the Holy One
\end{itemize}
}

\overviewsection[1]{%
\textbf{Verses 6-8: Cleansing and Commission}
\begin{itemize}
    \item A seraph cleanses Isaiah's lips with a coal from the altar
    \item Isaiah responds to God's call: "Here I am! Send me."
\end{itemize}
}

\overviewsection[0]{%
\textbf{Verses 9-13: The Hard Message}
\begin{itemize}
    \item God gives Isaiah a difficult message of judgment
    \item The people will hear but not understand until the land is desolate
\end{itemize}
}

\end{overview}

% Verses 1-5 - Chiastic Structure
\begin{chiasticoutline}[Isaiah 6:1-5]{.75em}{2em}

    \chiasticverselabel[A]{0}{
\versenum{1} In the year that \highlightblue{King} Uzziah died I saw the Lord \highlightgreen{sitting} upon a throne, high and lifted up; and the train of his robe filled the temple.
    }{Seen the King}

    \chiasticverselabel[B]{1}{
\versenum{2} Above him stood the seraphim. Each had six wings: with two he covered his face, and with two he covered his feet, and with two he flew.
\versenum{3} And one called to another and said: ``Holy, holy, holy is the LORD of hosts; the whole earth is full of his glory!''
\versenum{4} And the foundations of the thresholds shook at the voice of him who called, and the house was filled with smoke.
    }{Holy, Holy, Holy}

    \chiasticverselabel[A']{0}{
\versenum{5} And I said: ``Woe is me! For I am lost; for I am a man of unclean lips, and I \highlightgreen{dwell} in the midst of a people of unclean lips; for my eyes have seen the \highlightblue{King}, the LORD of hosts!''
    }{Seen the King}

\end{chiasticoutline}

% Bible Translation Comparison for verse 5a
\begin{biblecomparison}{Isaiah 6:5a}

\translation{ESV}{And I said: "Woe is me! For I am lost;"}

\translation{NASB}{"So I said, "Woe is me, for I am ruined!"}

\translation{KJV}{Then said I, Woe is me! for I am undone;}

\translation{NIV}{"Woe to me!" I cried. "I am ruined!"}

\translation{NET}{I said, "Too bad for me! I am destroyed,"}

\end{biblecomparison}

{\large\bfseries Isaiah's Response to Holiness}
{\vspace{1em}}

The variety in translation of Isaiah's response reveals the depth of his spiritual crisis. The ESV's "I am lost" (Hebrew: \textit{nidmeiti}) can mean destroyed, silenced, or undone. The NASB and NIV's "ruined" emphasizes total devastation, while the KJV's "undone" suggests being unraveled or coming apart. The NET's "destroyed" is perhaps most literal. All translations capture Isaiah's recognition that encountering God's holiness exposes his complete unworthiness and need for divine intervention.

{\vspace{1em}}

{\large\bfseries The Real King}
{\vspace{1em}}

King Uzziah had reigned for 52 years when he died, making him one of the longest-reigning kings of Judah. His death marked the end of an era of prosperity and stability. But in this moment of political uncertainty, Isaiah sees the true King - the LORD of hosts - seated on His eternal throne. This vision reminds us that earthly kings come and go, but the LORD reigns forever. See 2 Chronicles 26:16-20 for the account of Uzziah's downfall due to his pride.

{\vspace{1em}}
TODO: Add more details about King Uzziah and the political context.

{\large\bfseries Seraphim}
{\vspace{1em}}

The word "seraphim" comes from the Hebrew root meaning "to burn." These are the only creatures called seraphim in Scripture, and they are associated with purification and holiness. Interestingly, the same Hebrew word is used for the "fiery serpents" in Numbers 21:6, suggesting these beings are connected with God's purifying judgment.

{\vspace{1em}}
TODO: Add more details about the significance of seraphim.

{\large\bfseries Guarding the Thresholds}
{\vspace{1em}}

The shaking of the thresholds recalls the cherubim who guard the entrance to Eden after the fall (Genesis 3:24). Just as angels guarded the way to the tree of life, the seraphim here guard the threshold of God's holy presence. The shaking thresholds emphasize the awesome power and holiness of God that makes His presence both wonderful and terrible.

{\vspace{1em}}
TODO: Add more details about threshold symbolism and connection to Eden.

% Verses 6-8 - Biblical Outline
\begin{biblicaloutline}[Isaiah 6:6-8]
    
    \subsectionheader{Cleansing (6-7)}

    \begin{versesection}{2em}
        \versenum{6} Then one of the seraphim flew to me, having in his hand a burning coal that he had taken with tongs from the altar.

        \versenum{7} And he touched my mouth and said: ``Behold, this has touched your lips; your guilt is taken away, and your sin atoned for.''
    \end{versesection}
    
    \subsectionheader{Commission (8)}
    
    \begin{versesection}{2em}
        \versenum{8} And I heard the voice of the Lord saying, ``Whom shall I send, and who will go for us?'' Then I said, ``Here I am! Send me.''
    \end{versesection}

\end{biblicaloutline}

{\large\bfseries Actually, we'll end it there}
{\vspace{1em}}

TODO: Add notes for verses 6-8.

% Verses 9-13 - Biblical Outline
\begin{biblicaloutline}[Isaiah 6:9-13]
    
    \subsectionheader{The Hard Message (9-10)}

    \begin{versesection}{2em}
        \versenum{9} And he said, ``Go, and say to this people: `Keep on hearing, but do not understand; keep on seeing, but do not perceive.'

        \versenum{10} Make the heart of this people dull, and their ears heavy, and blind their eyes; lest they see with their eyes, and hear with their ears, and understand with their hearts, and turn and be healed.''
    \end{versesection}
    
    \subsectionheader{How Long? (11-13)}
    
    \begin{versesection}{2em}
        \versenum{11} Then I said, ``How long, O Lord?'' And he said: ``Until cities lie waste without inhabitant, and houses without people, and the land is a desolate waste,

        \versenum{12} and the LORD removes people far away, and the forsaken places are many in the midst of the land.

        \versenum{13} And though a tenth remain in it, it will be burned again, like a terebinth or an oak, whose stump remains when it is felled.'' The holy seed is its stump.
    \end{versesection}

\end{biblicaloutline}

{\vspace{2em}}
TODO: Add notes for verses 9-13.

\end{document}