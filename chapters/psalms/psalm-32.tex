\documentclass[11pt]{article}
\usepackage[margin=1in]{geometry}
\usepackage{../../styles/isaiah}

\begin{document}

\newpage
\begin{center}
{\Huge\bfseries Psalm 32}
\end{center}
\vspace{8em}

% Increase font size for the entire document
\large

% Psalm 32 - First Chiastic Structure (vv. 1-5)
\begin{chiasticoutline}[Psalm 32:1-5 – Confession and Forgiveness]{.99999em}{2em}

\chiasticverselabel[A]{0}{
A \sectionwordfootnote{Maskil}{musical term, maybe signaling an instructive/wisdom Psalm but most folks don't really know} of David.

\versenum{1} \highlightpurple{Blessed} is the one whose \highlightred{transgression} is \sectionwordfootnote{\highlightgreen{forgiven}}{literally 'lifted up' - compare with the heavy hand in v4},

\poetryline whose \highlightred{sin} is \highlightorange{covered}.

\versenum{2} \highlightpurple{Blessed} is the man against whom the \highlightblue{LORD} counts no \highlightred{iniquity},

\poetryline and in whose spirit there is no deceit.
}{}

\chiasticverse[B]{1}{
\versenum{3} For when I kept silent,

\poetryline my bones wasted away

\poetryline through my \sectionwordfootnote{groaning}{lit. 'roaring'} all day long.

\versenum{4} For day and night your hand was heavy upon me;

\poetryline my strength was \sectionwordfootnote{dried up}{lit. 'overturned'} as by the heat of summer. Selah
}

\chiasticverselabel[A']{0}{
\versenum{5} I acknowledged my \highlightred{sin} to you,

\poetryline and I did not \highlightorange{cover} my \highlightred{iniquity};

I said, "I will confess my \highlightred{transgressions} to the \highlightblue{LORD},"

\poetryline and you \highlightgreen{forgave} the \highlightred{iniquity} of my \highlightred{sin}. Selah
}{}

\end{chiasticoutline}

\newpage
% Psalm 32 - Second Chiastic Structure (vv. 6-11)
\begin{chiasticoutline}[Psalm 32:6-11 – Trust and Joy]{.99999em}{2em}

\chiasticverselabel[A]{0}{
\versenum{6} Therefore let \highlightyellow{everyone} who is \highlightaqua{godly}

\poetryline offer prayer to you at a time when you may be found;

surely in the rush of \sectionwordfootnote{great waters}{often a symbol of chaos and destruction},

\poetryline they shall not reach him.

\versenum{7} You are a hiding place for me;

\poetryline you preserve me from trouble;

\poetryline you \highlightsilver{surround} me with \highlightbrown{shouts of deliverance}. Selah
}{}

\chiasticverse[B]{1}{
\versenum{8} I will instruct you and teach you in the way you should go;

\poetryline I will counsel you with my eye upon you.

\versenum{9} Be not like a horse or a mule, without understanding,

\poetryline which must be curbed with bit and bridle,

\poetryline or it will not stay near you.
}

\chiasticverselabel[A']{0}{
\versenum{10} Many are the sorrows of the wicked,

\poetryline but steadfast love \highlightsilver{surrounds} the one who trusts in the \highlightblue{LORD}.

\versenum{11} Be glad in the \highlightblue{LORD}, and rejoice, O you righteous,

\poetryline and \highlightbrown{shout for joy}, \highlightyellow{all} you upright in heart!
}{}

\end{chiasticoutline}

\newpage

% Questions section
\begin{biblicaloutline}[Discussion Questions]


\begin{versesection}{2em}
\textbf{1.} How does this Psalm relate to Psalm 1?

\vspace{4em}

\textbf{2.} What does it mean to be blessed here and how do we use that today?

\vspace{4em}

\textbf{3.} What does the repeated use of the word "covering" signify here? Why is it important?

\vspace{4em}

\textbf{4.} Compare the silent vs shouting theme. What is leading David to go from silence in v3 to shouting in v11?

\vspace{4em}

\textbf{5.} What does v9 mean when it says that "it will not stay near you"?

\vspace{4em}

\textbf{6.} What is the close tie between "spiritual" and "physical" deliverance here? Do we view the line between these two similarly to David here?

\vspace{4em}

\textbf{7.} Do you have a culture of confession in your life, your home, your GC? If not, how does this Psalm encourage or convict you to grow in this area?
\end{versesection}

\end{biblicaloutline}

\end{document}
