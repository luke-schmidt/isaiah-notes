\documentclass[11pt]{article}
\usepackage[margin=1in]{geometry}
\usepackage{../../styles/isaiah}
\usepackage{../../styles/components/threeBoxGrid}

\begin{document}

% Overview of Isaiah 1-12 Chiastic Structure
\isaiahChiasticOverview{5}

\newpage

% Isaiah Context Grid
\threeBoxGrid{3}{Isaiah 9:8-10:34}{
    \textbf{\large Isaiah 9:8-21}

    \textit{Anger not Turned \\Away from Israel}
}{
    \textbf{\large Isaiah 10:1-4}

    \textit{Woe to Israel \& Anger \\not Turned Away}
}{
    \textbf{\large Isaiah 10:5-34}

    \textit{Woes and Therefores \\to Assyria}
}

% Overview of Isaiah 10:5-34
\begin{overview}{Isaiah 10:5-34 — Overview}

\overviewsection[0]{%
\textbf{Verses 5-15: \highlightred{Assyria} is coming}
}

\overviewsection[1]{%
\textbf{Verses 16-19: \highlightblue{Therefore \#1:} Yahweh will burn up \highlightred{Assyria} but a remnant will remain}
}

\overviewsection[2]{%
\textbf{Verses 20-23: A \highlightorange{remnant} of Israel will trust Yahweh}
}

\overviewsection[1]{%
\textbf{Verses 24-27: \highlightblue{Therefore \#2:} Don't be afraid of \highlightred{Assyria}}
}

\overviewsection[0]{%
\textbf{Verses 28-34: \highlightred{Assyria} is coming}
}

\end{overview}
\newpage

% Verses 5-15 - Assyria is coming
\begin{chiasticoutline}[Isaiah 10:5-15]{.99em}{2em}

\chiasticverselabel[A]{0}{
\versenum{5} Woe to \highlightred{Assyria}, the \highlightbrown{rod} of my anger;

\poetryline the \highlightbrown{staff} in their \highlightpurple{hands} is my fury!

\versenum{6} Against a godless nation I send him,

\poetryline and against the people of my wrath I command him,

to \sectionwordfootnote{take spoil and seize plunder}{shalal-shalal baz-baz – like Isaiah's son from ch. 8}

\poetryline and to tread them down like the mire of the streets.

\versenum{7} But he does not so intend,

\poetryline and his heart does not so think;

but it is in his heart to destroy,

\poetryline and to cut off nations not a few;
}{}

\chiasticverselabel[B]{2}{
\versenum{8} \highlightaqua{for he says:}

\poetryline "Are not my commanders all kings?

\versenum{9} Is not Calno like Carchemish?

\poetryline Is not Hamath like Arpad?

\poetryline Is not Samaria like Damascus?

\versenum{10} As my \highlightpurple{hand} has reached to the kingdoms of the idols,

\poetryline whose carved images were greater than those of Jerusalem and Samaria,

\versenum{11} shall I not do to Jerusalem and her idols

\poetryline as I have done to Samaria and her images?"
}{}

\chiasticverselabel[C]{4}{
\versenum{12} When the Lord has \sectionwordfootnote{finished}{Lit. "Cut off" – Motyer says it's a tailoring/weaving term. Like snipping off of a thread.} all his work on Mount Zion and on Jerusalem, he will punish the speech of the arrogant heart of the king of \highlightred{Assyria} and the boastful look in his eyes.
}{}

\chiasticverselabel[B']{2}{
\versenum{13} \highlightaqua{For he says:}

\poetryline "By the strength of my \highlightpurple{hand} I have done it,

\poetryline and by my wisdom, for I have understanding;

I remove the boundaries of peoples,

\poetryline and plunder their treasures;

\poetryline like a bull I bring down those who sit on thrones.

\versenum{14} My \highlightpurple{hand} has \sectionwordfootnote{found}{Reached} like a nest

\poetryline the \sectionwordfootnote{wealth of the peoples}{Compare with v10 what else His hand has reached to};

and as one gathers eggs that have been forsaken,

\poetryline so I have gathered all the earth;

and there was none that moved a wing

\poetryline or opened the mouth or chirped."
}{}

\chiasticverselabel[A']{0}{
\versenum{15} Shall the axe boast over him who hews with it,

\poetryline or the saw magnify itself against him who wields it?

As if a \highlightbrown{rod} should wield him who \highlightbrown{lifts it},

\poetryline or as if a \highlightbrown{staff} should \highlightbrown{lift him} who is not wood!
}{}

\end{chiasticoutline}

{\large\bfseries Cities Leading to Jerusalem (v9-11)}
\vspace{1em}

\begin{center}
\includegraphics[width=0.9\textwidth]{image.png}
\end{center}

The natural question after this list would be: Where to next?

\vspace{3em}
{\large\bfseries Why Does Assyria Care About Jerusalem's Idols?}
\vspace{1em}

Verse 10 raises an interesting question - why does Assyria compare the "kingdoms of the idols" and their "carved images" with those of Jerusalem and Samaria?

On one level, this could be about Assyria being an unwitting instrument of God, unknowingly showing His ultimate concern for true worship. God is using them to demonstrate that idolatry - whether in Damascus, Samaria, or Jerusalem - deserves judgment.

But there's probably something more happening here. Assyria is essentially declaring that they are the god Israel should now worship - that their military might proves their supremacy over all deities, including Yahweh. Their conquests have become their theology: "We conquered nations with greater gods than yours, so your God must be even weaker."

This connects to the larger theme of who truly holds power and sovereignty. Assyria mistakes permission for power, confusing God's instrumental use of them with their own inherent authority. They don't realize they're simply a tool in the divine hand.

\vspace{3em}
{\large\bfseries The "Fruit" of Assyria's Pride}
\vspace{1em}

Let's look at how different translations handle verse 12, which is the center point (C) of this chiastic structure:

\begin{biblecomparison}{Isaiah 10:12}
\translation{ESV}{When the Lord has finished all his work on Mount Zion and on Jerusalem, he will punish \textbf{the speech of the arrogant heart} of the king of Assyria and the boastful look in his eyes.}
\translation{NASB}{So it will be that when the Lord has completed all His work on Mount Zion and on Jerusalem, He will say, "I will punish \textbf{the fruit of the arrogant heart} of the king of Assyria and the glory of his haughty eyes."}
\translation{NIV}{When the Lord has finished all his work against Mount Zion and Jerusalem, he will say, "I will punish the king of Assyria for \textbf{the willful pride of his heart} and the haughty look in his eyes."}
\translation{KJV}{Wherefore it shall come to pass, that when the Lord hath performed his whole work upon mount Zion and on Jerusalem, I will punish \textbf{the fruit of the stout heart} of the king of Assyria, and the glory of his high looks.}
\translation{NET}{When the sovereign master finishes judging Mount Zion and Jerusalem, then I will punish the king of Assyria for \textbf{what he has proudly planned} and for the arrogant attitude he displays.}
\end{biblecomparison}

\vspace{1em}

The word "fruit" in the NASB and KJV is significant when we consider the garden imagery throughout Isaiah. What looks good in Assyria's eyes - their own wisdom, understanding, and accomplishments - is actually forbidden fruit.

Assyria is following in Adam's footsteps! They're reaching for knowledge and power that isn't theirs to take, trusting in their own understanding rather than acknowledging the true source of their success.

Like Adam, they see something that appears good and wise to their eyes, but in grasping for it, they're actually rebelling against God's authority. Will we trust in God's wisdom or our own?

% Verses 16-19 - Therefore #1: Yahweh will burn up Assyria but a remnant will remain
\begin{biblicaloutline}[Isaiah 10:16-19]

    \begin{versesection}{2em}
        \versenum{16} \highlightblue{Therefore} the Lord GOD of hosts
        \poetryline will send wasting sickness among his stout warriors,
        and under his glory a burning will be kindled,
        \poetryline like the burning of fire.
        \versenum{17} The light of Israel will become a fire,
        \poetryline and his Holy One a flame,
        and it will burn and devour
        \poetryline his thorns and briers in one day.
        \versenum{18} The glory of his forest and of his fruitful land
        \poetryline the \highlightyellow{LORD} will destroy, both soul and body,
        \poetryline and it will be as when a sick man wastes away.
        \versenum{19} The \highlightorange{remnant} of the trees of his forest will be so few
        \poetryline that a child can write them down.
    \end{versesection}

\end{biblicaloutline}

\vspace{3em}
{\large\bfseries Assyria's Purifying Fire - A Parallel to Isaiah 6 and Isaiah 4}
\vspace{1em}

What's happening to Assyria here looks remarkably similar to what Isaiah experienced in chapter 6 and what Israel goes through in chapter 4. In Isaiah 6, the prophet encounters purifying fire from the altar that cleanses him and prepares a remnant for service. In Isaiah 4:2-4, the "branch of the LORD" will be beautiful, and those who remain in Jerusalem will be called holy after "the Lord shall have washed away the filth of the daughters of Zion and cleansed the bloodstains of Jerusalem from its midst by a spirit of judgment and by a spirit of burning."

Now in chapter 10, Assyria faces the same pattern: "The light of Israel will become a fire, and his Holy One a flame" (v. 17). Just like Israel, Assyria will be burned down to a remnant - "so few that a child can write them down" (v. 19).

Isaiah's prophetic vision treats Assyria with the same framework he uses for Israel! The instrument of judgment will itself be judged, purified, and reduced to a remnant. God's dealings with the nations mirror His dealings with His own people. Both face fire, both have remnants, both are subject to the same holy standards.

% Verses 20-23 - A remnant of Israel will trust Yahweh
\begin{biblicaloutline}[Isaiah 10:20-23]

    \begin{versesection}{2em}
        \versenum{20} In that day the \highlightorange{remnant} of Israel and the survivors of the house of Jacob will no more lean on him who struck them, but will lean on the \highlightyellow{LORD}, the Holy One of Israel, in truth.
        \versenum{21} A \highlightorange{remnant} will return, the \highlightorange{remnant} of Jacob, to the mighty God.
        \versenum{22} For though your people Israel be as the sand of the sea, only a \highlightorange{remnant} of them will return. Destruction is decreed, overflowing with righteousness.
        \versenum{23} For the Lord \highlightyellow{GOD of hosts} will make a full end, as decreed, in the midst of all the \sectionwordfootnote{earth}{Land}.
    \end{versesection}

\end{biblicaloutline}

\vspace{3em}
{\large\bfseries The Abrahamic Promise Still Stands}
\vspace{1em}

Verse 22 contains a crucial reference: "For though your people Israel be as the sand of the sea, only a remnant of them will return." This directly echoes God's promise to Abraham in Genesis 22:17 - "I will surely bless you, and I will surely multiply your offspring as the stars of heaven and as the sand that is on the seashore."

In their coming exile and captivity, Isaiah isn't declaring that God's covenant with Abraham is null and void. The promise still stands - there will still be a people who are fruitful and multiply. But Israel cannot use their chosen status as a crutch. Being Abraham's descendants doesn't exempt them from obedience and reverence toward God.

This connects powerfully to Paul's wrestling with this same tension in Romans 9-11. The gifts and calling of God are irrevocable, yet faith and obedience still matter. God's faithfulness to His promises doesn't mean His people can presume upon His grace. The remnant theology here anticipates Paul's "remnant according to the election of grace" (Romans 11:5). God will preserve a people, but being ethnically descended from Abraham guarantees nothing apart from faith.

% Verses 24-27 - Therefore #2: Don't be afraid of Assyria
\begin{biblicaloutline}[Isaiah 10:24-27]

    \begin{versesection}{2em}
        \versenum{24} \highlightblue{Therefore} thus says the Lord \highlightyellow{GOD of hosts}: "O my people, who dwell in Zion, be not afraid of the \highlightred{Assyrians} when they strike with the \highlightbrown{rod} and \highlightbrown{lift up their staff} against you as the Egyptians did.
        \versenum{25} For in a very little while my fury will come to an end, and my anger will be directed to their destruction.
        \versenum{26} And the \highlightyellow{LORD of hosts} will wield against them a whip, as when he struck Midian at the rock of Oreb. And his \highlightbrown{staff} will be over the sea, and he will \highlightbrown{lift it} as he did in Egypt.
        \versenum{27} And in that day his burden will depart from your shoulder, and his yoke from your neck; and the yoke will be broken because of the \sectionwordfootnote{fat}{or "oil" – confusing in the Hebrew as well. Could be a picture of an ox so fat from such little work that the yoke breaks}."
    \end{versesection}

\end{biblicaloutline}

\vspace{3em}
{\large\bfseries The Rod Reversed: God's Weapon Turned}
\vspace{1em}

Tracking the repeated words of "rod" and "staff" here can be helpful. In verse 5, Assyria is "the rod of my anger" and "the staff in their hands is my fury." The rod and staff are in God's hands, being wielded through Assyria against Israel.

But now in verse 26, everything reverses: "the LORD of hosts will wield against them a whip... And his staff will be over the sea, and he will lift it as he did in Egypt."

The rod and staff that was Assyria in God's hand now becomes a different rod and staff - this time wielded \textit{against} Assyria. The instrument becomes the target. The weapon becomes the enemy.

% Verses 28-34 - Assyria is coming
\begin{biblicaloutline}[Isaiah 10:28-34]

    \begin{versesection}{2em}
        \versenum{28} He has come to Aiath;
        \poetryline he has passed through Migron;
        \poetryline at Michmash he stores his baggage;
        \versenum{29} they have crossed over the pass;
        \poetryline at Geba they lodge for the night;
        Ramah trembles;
        \poetryline Gibeah of Saul has fled.
        \versenum{30} Cry aloud, O daughter of Gallim!
        \poetryline Give attention, O Laishah!
        \poetryline O poor Anathoth!
        \versenum{31} Madmenah is in flight;
        \poetryline the inhabitants of Gebim flee for safety.
        \versenum{32} This very day he will halt at Nob;
        \poetryline he will shake his \sectionwordfootnote{\highlightpurple{fist}}{Hand}
        \poetryline at the mount of the daughter of Zion,
        \poetryline the hill of Jerusalem.
        \versenum{33} Behold, the Lord \highlightyellow{GOD of hosts}
        \poetryline will lop the boughs with terrifying power;
        the great in height will be hewn down,
        \poetryline and the lofty will be brought low.
        \versenum{34} He will cut down the thickets of the forest with an axe,
        \poetryline and Lebanon will fall by the Majestic One.
    \end{versesection}

\end{biblicaloutline}

\vspace{3em}
{\large\bfseries Poetry of Threat, Not Literal Warpath}
\vspace{1em}

The towns listed in verses 28-32 are all located at the northern border of Judah - situated between the northern and southern kingdoms. This geographic detail is significant because historically, when Assyria actually invaded, they came up from the \textit{south} of Jerusalem (see Isaiah 36:1-2, where Sennacherib comes from Lachish, which is southwest of Jerusalem).

This passage isn't giving us a literal military campaign route or a step-by-step warpath. Rather, it's poetry that describes the imminent threat of Assyria. The crescendo builds with each town mentioned, getting closer and closer to Jerusalem, creating a sense of mounting dread. By verse 32, they're at Nob - "he will shake his fist at the mount of the daughter of Zion, the hill of Jerusalem."

This is poetic geography serving theological purpose - showing how close the threat is, how real the danger feels, how the noose is tightening around God's holy city. It's meant to evoke fear and urgency, not to be a historical chronicle of troop movements.

\vspace{3em}
{\large\bfseries Israel and the Nations: The Same Pattern}
\vspace{1em}

Throughout Isaiah, there's a consistent pattern: Israel and the nations are put on analogy. They share the same problem (rebellion against God), face the same consequences (judgment and exile), and remarkably, even share the same hope (restoration through a remnant).

Look ahead to Isaiah 19:19-25, where Egypt and Assyria will be blessed alongside Israel: "In that day Israel will be the third with Egypt and Assyria, a blessing in the midst of the earth, whom the LORD of hosts has blessed, saying, 'Blessed be Egypt my people, and Assyria the work of my hands, and Israel my inheritance.'"

This is radical theology. God's standards apply to everyone. His judgment falls on both His people and the nations. And His mercy extends to both as well. The same God who disciplines Israel with the rod of Assyria will discipline Assyria. The same God who preserves a remnant of Israel will preserve a remnant of Assyria. This universalizing of God's character - His justice and His mercy applying to all peoples - runs throughout the entire book of Isaiah.

\vspace{3em}
{\large\bfseries Beyond the Victor's Gods}
\vspace{1em}
\begin{quote}
\textit{"In 10:5-34 the amazing doctrine is stated that the gods are not necessarily on the side of the victors and that defeat for us is not defeat for God. Although this understanding has been a part of the western philosophy of history for sixteen hundred years, it is still difficult for us to translate it into feelings. Like our ancient Near Eastern forebears we instinctively believe that the victor's gods are God indeed and that the defeated's god is unmasked as a charlatan. Against this, Isaiah envisions a God who is not the prisoner of history, who is not the alter ego of either victor or vanquished, but who guides all events to an outcome in keeping with his own joyous and beneficent plan. All are under his hand. He is neither the possession nor the manifestation of any of his creatures. This is the doctrine of transcendence, a truth of unparalleled importance for life and understanding. Isaiah was not the first to formulate it (contra the older evolutionary theory of Israelite religion), for it is implicit in the first three commandments of the Decalog, but it may be fair to say that he was the first to apply it to the historical process in a thoroughgoing way."}\\\\
\hfill --- John N. Oswalt, \textit{The Book of Isaiah, Chapters 1-39}
\end{quote}

\end{document}
