\documentclass[11pt]{article}
\usepackage[margin=1in]{geometry}
\usepackage{../../styles/isaiah}


\begin{document}


\isaiahOverviewGrid{3}

{\vspace{15em}}
% Isaiah 2:6-22 Study
\begin{chiasticoutline}[Isaiah 2:6-22 – The Day of the LORD]{.99999em}{2em}

\chiasticverselabel[A]{0}{
\versenum{6} For you have rejected your people,

\poetryline the house of Jacob,

because they are full of things from the east

\poetryline and of fortune-tellers like the Philistines,

\poetryline and they strike hands with the children of foreigners.

\versenum{7} Their land is filled with silver and gold,

\poetryline and there is no end to their treasures;

their land is filled with horses,

\poetryline and there is no end to their chariots.

\versenum{8} Their land is filled with \highlightred{idols};

\poetryline they \highlightpurple{bow down} to the work of their hands,

\poetryline to what their own fingers have made.

\versenum{9} So man is humbled,

\poetryline and each one is brought low—

\poetryline do not forgive them!

\versenum{10} \boldbrown{Enter into the rock}

\poetryline and hide in the dust

\highlightblue{from before the terror of the LORD}

\poetryline \highlightblue{and from the splendor of his majesty}.
}{idols, rocks, and terror}

\chiasticverse[B]{1}{
\boldgreen{\versenum{11} The haughty looks of man shall be brought low,}

\boldgreen{\poetryline and the lofty pride of men shall be humbled,}

\boldgreen{\poetryline and the LORD alone will be exalted in that day.}
}

\chiasticverse[C]{2}{
\versenum{12} For the LORD of hosts has a day

\poetryline against all that is proud and lofty,

\poetryline against all that is lifted up—and it shall be brought low;

\versenum{13} against all the cedars of Lebanon,

\poetryline lofty and lifted up;

\poetryline and against all the oaks of Bashan;

\versenum{14} against all the lofty mountains,

\poetryline and against all the lifted up hills;

\versenum{15} against every high tower,

\poetryline and against every fortified wall;

\versenum{16} against all the ships of Tarshish,

\poetryline and against all the beautiful craft.
}

\chiasticverse[B]{1}{
\boldgreen{\versenum{17} And the haughtiness of man shall be humbled,}

\boldgreen{\poetryline and the lofty pride of men shall be brought low,}

\boldgreen{\poetryline and the LORD alone will be exalted in that day.}
}

\chiasticverselabel[A]{0}{
\versenum{18} And the \highlightred{idols} shall utterly pass away.

\versenum{19} And people shall \boldbrown{enter the caverns of the rocks}

\poetryline and the holes of the ground,

\highlightblue{from before the terror of the LORD}

\poetryline \highlightblue{and from the splendor of his majesty},

\poetryline when he rises to terrify the earth.

\versenum{20} In that day mankind will cast away

\poetryline their \highlightred{idols} of silver and their \highlightred{idols} of gold,

which they made for themselves to \highlightpurple{worship},

\poetryline to the moles and to the bats,

\versenum{21} to \boldbrown{enter the caverns of the rocks}

\poetryline and the clefts of the cliffs,

\highlightblue{from before the terror of the LORD}

\poetryline \highlightblue{and from the splendor of his majesty},

\poetryline when he rises to terrify the earth.

\versenum{22} Stop regarding man

\poetryline in whose nostrils is breath,

\poetryline for of what account is he?
}{idols, rocks, and terror x2}

\end{chiasticoutline}


\vspace{3em}
{\large\bfseries Money, Tanks, and Idols}
\vspace{1em}

In verses 7-8 we see, in a nice poetic form, what the land of Jacob is really filled with
and what kinds of things they're placing their hope in – money (silver and gold), military prowess (horses and chariots), and idols.

{\vspace{1em}}

It's a stark contrast to verse 4 where Israel is destroying their military tools into productive equipment for the communities benefit.

This is where the "real" Israel is at.


\vspace{3em}
{\large\bfseries Don't Forgive Them?}
\vspace{1em}

So what's going on in verse 9 though? How will we be able to get to v1-5 if there's no forgiveness for the house of Jacob?

{\vspace{1em}}

\begin{quote}
\textit{"...It seems to express the prophet's deep despair over his people's condition. He seems almost afraid that God might relent and, in violation of his own justice, forget their heinous sins. As such, this phrase exposes the problem of sin. It cannot be simply forgotten, it must be punished; otherwise, the whole chain of cause and effect upon which the world is built would be broken. But the punishment for playing god can be no less than banishment from God, a denial of God's purpose in creating human beings-fellowship with himself. What might the solution be? Chs. 1-39 never really answer the question. Only in the second part of the book does the answer come (43:1-7; 44:21-22; 52:7; 53:12;
59:15-21; 53:1-6). But whatever the answer, it is not to be found in acting as if the sin had not been committed."}\\\\
\hfill --- John Oswalt, \textit{The Book of Isaiah, Chapters 1-39}
\end{quote}

\hebrewword{Forgive}{נָשָׂא}{na.sa}{"to raise" or "to lift up"}

The word "na.sa" is in direct contrast to the bowing down to the idols they're worshiping and the whole theme of this section of bringing low that which is high.

\begin{thesauce}
\sauceitem{Why is God against the trees? Is this a metaphore for the people? Is this because this is what they were worshiping back in chapter 1?}

\sauceitem{The "Ships of Tarshish" have a significant role in the Hebrew Bible. Yahewh "bringing them low" has a lot more significance than just "destroying ships"}

\sauceitem{There's probably more we could do with the Tower of Babel imagery here as well.}

\end{thesauce}

\end{document}